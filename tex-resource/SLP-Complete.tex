\documentclass[12pt]{article}
\usepackage[utf8]{inputenc}

\usepackage[T1]{fontenc}
\usepackage{amsmath,amsfonts,amssymb}
% \usepackage{geometry, hyperref, fancyhdr, titlesec, setspace, abstract, indentfirst, booktabs, multirow, array, caption, subcaption, fontspec, enumitem}
\usepackage{geometry, hyperref, fontspec, enumitem, graphicx, subfig, gensymb, abstract, fancyhdr}

\setlength{\headheight}{14.5pt}
\usepackage[dvipsnames]{xcolor}



\setmainfont{Georgia}


% Page setup
\pagestyle{fancy}
\fancyhf{}
\addtolength{\headwidth}{\marginparwidth}
\fancyhead[R]{\thepage}
\fancyhead[L]{\leftmark}
\renewcommand{\headrulewidth}{1pt}

\geometry{
    a4paper,
    left=2.0cm,
    right=2.0cm,
    top=3.0cm,
    bottom=2.0cm
}

% Hyperref setup
\hypersetup{
    colorlinks=true,
    linkcolor=RoyalBlue,
    filecolor=magenta,      
    urlcolor=RoyalBlue,
    pdftitle={Laboratory Studies of Cosmology-Inspired Defect Dynamics in Liquid Crystals},
    pdfauthor={Anik Mandal},
    bookmarksopen=true,
    bookmarksnumbered=true
}

% Custom commands
\newcommand{\email}[1]{\texttt{#1}}
\newcommand{\university}{Ashoka University}
\newcommand{\location}{Sonipat, India}


\begin{document}

% Title page
\begin{titlepage}
    \centering
    \vspace*{2cm}
    {\huge\bfseries Laboratory Studies of Cosmology Inspired Defect Dynamics in \\Liquid Crystals\par}
    \vspace{2cm}
    {\Large Semester Long Project Report\par}
    \vspace{1.5cm}
    {\large
    \textbf{Anik Mandal}$^{1,*}$ \\[0.5em]
    \begin{tabular}{r l}
        SLP Guide &:  Prof. Pramoda Kumar$^{1}$ \\
        SLP Co-guide &: Ankit Gupta$^{1}$ \\
        PhD Supervisor &:  Prof. Somak Raychaudhury$^{1}$ \\[1em]
    \end{tabular}

  
    $^1$\university, \location \\
    $^{*}$Email: \href{mailto:anik.mandal_phd24@ashoka.edu.in}{anik.mandal\_phd24@ashoka.edu.in}
    }\\
    \vspace{2cm}
    \includegraphics[width=0.2\linewidth]{tex-resource/ashoka-logo.png}
    \vfill
    {\large \today\par}
\end{titlepage}

% Abstract
\begin{abstract}
This project report presents a systematic investigation of cosmology-inspired defect dynamics in nematic liquid crystals (NLCs). We employ a combined experimental and computational approach to study the formation and evolution of topological defects under applied electric fields. Nematic liquid crystals serve as valuable laboratory analogues for cosmological phenomena, allowing us to observe and quantify defect dynamics on accessible time and length scales. In this work, we examine the Freedericksz transition and subsequent defect formation in a well-characterized MBBA (4-cyano-4'-n-pentylbiphenyl) liquid crystal cell under various voltage conditions. High-speed optical microscopy data are processed using advanced image analysis techniques including ridge detection and morphological operations to extract quantitative measures of defect density. Our analysis focuses on late-time coarsening dynamics, where we observe scaling behavior consistent with theoretical predictions from the one-scale model of defect network evolution. We present experimental results from multiple voltage configurations (10V, 20V, 40V, and 60V) and demonstrate that the observed scaling exponents agree with the predicted $\rho(t) \propto 1/t$ power law, providing experimental validation of fundamental concepts in nonequilibrium statistical mechanics and cosmological defect theory.
\end{abstract}

% Table of Contents
\setcounter{tocdepth}{2}
\tableofcontents
\newpage

% Main content
\section{Introduction}

\subsection{Symmetry-breaking in early Universe}

Symmetry-breaking phase transitions play a fundamental role in modern physics, particularly in understanding the evolution of the early universe and the formation of cosmological structures. In the context of particle physics and cosmology, symmetry-breaking transitions occur as the universe cools following the Big Bang, with increasingly more complex symmetries becoming relevant at lower energies. When a system undergoes such a phase transition, the underlying symmetry of the Lagrangian is reduced, resulting in a phenomenon known as spontaneous symmetry breaking. This process has profound consequences: the vacuum acquires a non-trivial expectation value, and the fields acquire preferred orientations in internal space.

A key characteristic of symmetry-breaking transitions is the existence of a potential energy function (often referred to as a Higgs potential in particle physics contexts) that transitions from a parabolic to a Mexican-hat configuration. Below the critical temperature, the system can no longer remain at the potential minimum of zero field value; instead, it must settle into one of several equivalent minima arranged on a manifold. The set of all such equivalent minimum-energy configurations is called the vacuum manifold. The topology of this vacuum manifold—specifically its topological properties characterized by homotopy groups—determines which types of topological defects can exist in the broken-symmetry phase.

\subsection{Cosmological defects dynamics}

Topological defects are spatially extended objects that arise naturally in theories with spontaneously broken symmetries. These defects represent regions where the order parameter cannot be continuously deformed to a uniform configuration without encountering singularities. The classification of topological defects depends on the homotopy group structure of the vacuum manifold: domain walls correspond to $\pi_0$ nontrivial topology, cosmic strings correspond to $\pi_1$ nontrivial topology, monopoles correspond to $\pi_2$ nontrivial topology, and textures correspond to $\pi_3$ nontrivial topology.

The dynamics of cosmological defect networks have been extensively studied through both analytical and numerical approaches. Following the Kibble mechanism, which describes the production of defects during phase transitions, defect networks enter a self-similar scaling regime where the characteristic length scale evolves according to well-defined power laws. The network coarsens through a combination of processes: direct annihilation of opposite-sign defects, loss of length into small loops, and intercommutation events (for strings) where crossing defects exchange connections. A crucial simplification emerges from the one-scale model, which posits that at all times the defect network can be characterized by a single length scale $\xi$ that represents both the typical radius of curvature of defects and their typical separation. Under conditions where viscous friction dominates the dynamics, the scaling solution predicts a defect density that evolves as $\rho(t) \propto 1/t$.

\subsection{Kibble-Zurek mechanism}

The Kibble-Zurek mechanism describes how topological defects are produced during symmetry-breaking phase transitions. The mechanism rests on three key insights: (1) during a rapid phase transition, causality prevents perfect synchronization of the order parameter orientation across distances exceeding the causal horizon, known as the Kibble length; (2) the system is divided into domains, each of which independently chooses a random vacuum state; (3) at the boundaries between domains with incompatible vacuum choices, defects must form to mediate the transition.

Originally proposed by Zurek in 1985, the Kibble-Zurek mechanism was suggested as a framework for testing cosmological predictions in laboratory systems. Zurek envisioned using a phase transition in superfluid helium-4 as an experimental test, but technical difficulties prevented successful implementation. The key insight of the mechanism is that the predicted defect density at the moment of quench depends on the quench rate: faster quenches produce more defects because the causally-connected region (Kibble length) is smaller. This relationship between quench dynamics and defect formation provides a direct connection between microphysical parameters and observable defect densities.

\subsection{Project objectives}

The primary objectives of this semester-long project are to: (1) experimentally investigate the formation and evolution of topological defects in nematic liquid crystals under controlled electric field conditions; (2) develop and validate computational methods for quantitative analysis of defect structures in optical microscopy data; (3) characterize the Freedericksz transition in our experimental cell and understand the threshold behavior for defect nucleation; (4) analyze defect dynamics across multiple voltage regimes to extract scaling laws and compare with theoretical predictions; and (5) establish the nematic liquid crystal system as a reliable laboratory analogue for studying cosmologically relevant defect physics. Through these objectives, we aim to bridge fundamental concepts in cosmology and particle physics with observable phenomena in a well-controlled condensed matter system.

\section{Experimental Setup}

\subsection{Nematic liquid crystals and their properties}

Nematic liquid crystals represent an intermediate phase of matter between isotropic liquids and crystalline solids. In the nematic phase, the constituent rod-shaped molecules exhibit long-range orientational order while maintaining positional disorder characteristic of liquids. The nematic-isotropic phase transition, understood through Onsager theory, occurs due to competition between positional entropy (favoring random molecular locations) and orientational entropy (favoring random molecular orientations). At high concentrations or low temperatures, orientational order dominates, and the nematic phase becomes thermodynamically stable.

The order parameter in a nematic liquid crystal is represented by the director field $\mathbf{n}(\mathbf{r})$, a unit vector describing the local preferred molecular orientation. Since the rod-shaped molecules are apolar (oriented along their long axis), the director is defined up to a sign: $\mathbf{n} \equiv -\mathbf{n}$. This equivalence means that the configuration space is the real projective plane $\mathbb{RP}^2$, or equivalently the two-sphere with antipodal points identified. The topology of this vacuum manifold is crucial for determining which defects can exist: $\pi_0(\mathbb{RP}^2) = \mathbb{Z}_2$ (domain walls), $\pi_1(\mathbb{RP}^2) = \mathbb{Z}$ (strings), and $\pi_2(\mathbb{RP}^2) = \mathbb{Z}$ (monopoles and textures).

In our experiments, we work with MBBA (4-methoxy benzylidene 4-n-butylaniline), a thermotropic nematic liquid crystal with a nematic-isotropic transition temperature of approximately 40°C. MBBA is chosen for its favorable optical properties, stable nematic phase at room temperature, and well-characterized physical parameters. The elastic properties of the nematic phase are governed by the Frank free energy density, which contains three types of elastic deformations: splay (divergence of the director), twist (rotational gradient), and bend (curvature). For MBBA, the elastic constants are approximately $K_1 = 1.15 \times 10^{-6}$ dyne, $K_2 = 0.6 \times 10^{-6}$ dyne, and $K_3 = 1.55 \times 10^{-6}$ dyne at 25°C, with different elastic moduli reflecting the anisotropic nature of molecular interactions.

\subsection{Homeotropic alignment and boundary conditions}

The boundary conditions imposed on the director at the liquid crystal-surface interface dramatically influence defect formation and evolution. In homeotropic alignment, the director is perpendicular (normal) to the surface, with molecules standing on end relative to the confining plates. This alignment is achieved through surface treatment with specific surfactants such as DMOAP (dimethyl octadecyl ammonium chloride), which promotes vertical molecular orientation through electrostatic and hydrophobic interactions.

Homeotropic alignment is particularly advantageous for our defect dynamics studies because it provides a well-defined initial state: in the absence of external fields, the director is uniformly perpendicular to the cell surfaces throughout the volume. When an electric field is subsequently applied parallel to the surfaces, it induces a competition between the anchoring energy (which tends to maintain homeotropic alignment) and the electric field energy (which tends to reorient the director). This competition leads to a threshold phenomenon—the Freedericksz transition—where beyond a critical field strength, the uniform homeotropic state becomes unstable and the director begins to distort.

\subsection{Preparing empty cell}

Preparation of liquid crystal cells follows a series of well-established, standardized steps. The standard workflow is outlined below.

\subsubsection*{Patterning:}
ITO-coated glass slides are cut and patterned to create defined electrode regions. The glass is placed on a cutting mat and the non-coated side is cut to 4 cm × 2 cm dimensions. A multimeter verifies the coated side. The coated surface is then marked with 1 cm wide cello tape, leaving 0.5 cm gaps at both ends. This protects the desired conducting area.

\subsubsection*{Etching:}
In a fume hood, concentrated HCl and zinc powder are added to a beaker containing the patterned glass (coated side up). The mixture etches the exposed ITO for 30 minutes, removing it from unprotected areas while leaving the tape-covered region intact. After etching, the glass is carefully removed with tweezers and rinsed thoroughly with distilled water and a dilute base to neutralize all acid residue. The glass is dried and transferred to the cleaning stage.

\subsubsection*{Post etching treatment:}
 The etched glass is cut vertically down the middle to produce two symmetric slabs. Both slabs are scrubbed with soap and isopropyl alcohol (IPA) to remove loose particles. They are then placed in a petri dish with a soap-water mixture and sonicated for 15 minutes at 40–50°C. After sonication, the slabs are rinsed with fresh IPA and dried with nitrogen gas, ensuring no residue remains.

\subsubsection*{Applying surfactant (DMOAP):}
0.2\% DMOAP surfactant solution is prepared by mixing 160 μL DMOAP with 80 mL distilled water. Both glass slabs are held vertically, with the coated sides facing the same direction, and dipped into the surfactant solution for 5 minutes. The surfactant forms a thin layer that promotes homeotropic (vertical) alignment. The slabs are rinsed with distilled water and dried with nitrogen gas.

\subsubsection*{Cell assembly:}
Both treated slabs are assembled as a sandwich with spacer beads (23 μm diameter) placed 1 cm apart around the edges. The spacers create a uniform gap between the glass plates. The assembled cell is pressed moderately for 30 minutes to stabilize the spacers. The cell edges are then sealed with epoxy and allowed to cure thoroughly for at least a day to ensure a strong bond. Once the epoxy is fully set, electrical wires are connected to the ITO electrodes by carefully soldering with indium.

\subsection{Capacitance-based cell thickness measurement}

The thickness of the liquid crystal cell is determined using a capacitance measurement method. The cell acts as a parallel plate capacitor, where the measured capacitance (C) is related to the thickness (d) by the formula:
\[
d = \frac{\epsilon_0\epsilon_r A}{C}
\]
with $\epsilon_0$ being the vacuum permittivity, $\epsilon_r$ the relative permittivity ($\approx 1$ for air), and $A$ the effective electrode area. A precision LCR meter is used to measure the cell's capacitance, and to accurately calculate the effective area, only the region of overlap between the patterned ITO electrodes is considered, excluding the edge regions and spacers. The measured thickness of the cell is $31.09 \pm 0.44$ μm. This measurement is critical for subsequent analysis, as the cell thickness enters directly into calculations of defect density (defect length per unit volume).

\subsection{Preparing homeotropic cell}

The assembled LC cell is placed on a temperature control device programmed with a three-stage thermal protocol: 
\begin{enumerate}
    \item heat to $50\degree C$ at a rate $20\degree C/\text{min}$ and hold for 10 mins,
    \item cool to $36\degree C$ at a rate $1\degree C/\text{min}$ and hold for 5 mins,
    \item cool to  $25\degree C$ at a rate $0.1\degree C/\text{min}$ and hold.
\end{enumerate}

Once the device reaches $55\degree C$, a small drop of MBBA liquid crystal is carefully placed on the cell edge using a lab spatula. As the MBBA warms and becomes transparent, it is gently pushed toward the cell center, where capillary action draws it throughout the cell gap. The lid is closed, and the cell is left undisturbed for at least 24 hours during the slow cooling protocol, allowing the liquid crystal to reach the nematic phase and establish uniform homeotropic alignment with minimal defects. The slow cooling at the final stage is particularly important: a rate of 0.1°C/min allows the liquid crystal molecules sufficient time to reorient and anneal into the lowest-energy homeotropic configuration. Upon completion, the cell is inspected under crossed polarizers to verify uniform alignment and is then ready for experimental use or stored in a sealed container at room temperature.

\subsection{Optical microscopy system}

Electrodes from the function generator, which delivers an RMS AC sine wave voltage ranging from 10V to 60V at 10 kHz via an A800DI high voltage linear amplifier, were connected to the cell. At a controlled temperature of 27.5°C, dynamic data of defect formation were recorded under 10× magnification for RMS voltages of 10V, 20V, 40V, and 60V. For each voltage, the experiment was repeated using three different optical configurations:
\begin{itemize}
\item Polarized illumination without an analyzer,
\item Polarized illumination with analyzer at a relative angle of 0°,
\item Polarized illumination with analyzer at a relative angle of 90°.
\end{itemize}

The choice of AC voltage at 10 kHz is deliberate: this frequency is above the ionic relaxation frequency of MBBA, avoiding complications from ionic currents that would dominate at lower frequencies and obscure the underlying liquid crystal physics. The 10× magnification provides a field of view approximately 1.4 mm × 1.0 mm, sufficient to capture the evolution of defect networks with adequate spatial resolution (~1 μm per pixel) for reliable image analysis. The three optical configurations provide complementary information: crossed polarizers (0° analyzer angle) maximize contrast for defect visualization, while the other configurations provide alternative contrast mechanisms useful for resolving ambiguous features.

\section{Computational Framework}

\indent Captured frame images are subsequently processed through a series of image analysis operations using the Python Sci-kit Image \cite{VanDerWalt2014Scikit-image:Python} package. The detailed workflow and framework \footnote{Framework scripts, notebooks and analysis results can be found here: \href{https://github.com/mandal-anik10/SLP-CosmoWithLCs}{https://github.com/mandal-anik10/SLP-CosmoWithLCs}\\
\textbf{Clone:} \$ git clone https://github.com/mandal-anik10/SLP-CosmoWithLCs.git} 
for this analytical pipeline are described below.

\subsection{Pre-processing and noise reduction}

In the first step of image processing, we implement a $3 \times 3$ median filter to reduce noise and improve the overall quality of the image. The median filter is a non-linear filtering technique that replaces each pixel value with the median of the intensity values within its $3 \times 3$ neighborhood. Specifically, for every pixel, a $3 \times 3$ window is centered on it, and the nine pixel values within this window are sorted in ascending order. The median value from this sorted list is then assigned to the central pixel, effectively suppressing impulsive noise such as salt-and-pepper noise while preserving important image details. Unlike linear filters that tend to blur edges by averaging pixel values, the median filter maintains sharp boundaries and fine structures, making it particularly effective for applications where edge preservation is crucial. This property is essential in our application since liquid crystal defects appear as sharp linear structures that must be preserved for accurate subsequent analysis.

\begin{figure}[h]
    \centering
    \includegraphics[width=1.0\linewidth]{figs/test-framework-2s_40V-S012.png}
    \caption{Images after $3 \times 3$ median noise filter and Meijering ridge detection. The sample image is a single frame from 40V $10 \times$ without analyzer dynamical data.}
    \label{fig:test_fw_012}
\end{figure}

\subsection{Ridge detection operation}

For ridge detection, we implement the Meijering ridge detection filter, a modified Hessian-based approach specifically designed to detect elongated, tubular structures in noisy images. This method computes the eigenvalues and eigenvectors of a modified Hessian matrix at each pixel, where the local principal ridge directions are determined by the eigenvectors of the matrix computed from intensity values in the local neighborhood. The Meijering filter assigns a ``neuriteness'' or ``vesselness'' measure $\Phi(x)$ to each pixel according to 
\[
\Phi(x) = \frac{\lambda(x)}{\lambda_{\text{min}}} \quad \text{for} \quad \lambda(x) \geq 0, \quad \text{and} \quad \Phi(x) = 0 \quad \text{otherwise}
\]
where $\lambda$ represents the larger eigenvalue in magnitude and $\lambda_{\text{min}}$ is the minimum eigenvalue across all pixels. The eigenvector corresponding to the smaller absolute eigenvalue indicates the longitudinal direction of the ridge, effectively suppressing responses to first-order structures such as background intensity discontinuities while enhancing continuous, filamentous features. This multi-scale approach proves superior to basic Hessian or Laplacian-based detectors for identifying nematic liquid crystal strings, as it maintains connectivity across regions of varying contrast and width while reducing false detections from noise artifacts. The ridge detection filter operates at multiple scales, allowing it to detect defects ranging from approximately 2 to 20 pixels in width.

\subsection{Calculating string density}

\subsubsection{Hysteresis thresholding}

To distinguish true ridge structures from spurious noise artifacts in the ridge-detected images, we apply hysteresis thresholding with two carefully selected thresholds, $T_{\text{high}} = 0.2$ and $T_{\text{low}} = 0.08$, following the methodology established in the Canny edge detection framework. In this dual-threshold approach, pixels with ridge metric values above $T_{\text{high}}$ are immediately classified as definite defect pixels (strong edges), while those below $T_{\text{low}}$ are rejected as background. Critically, pixels with values between $T_{\text{low}}$ and $T_{\text{high}}$ (weak edge pixels) are retained only if they remain connected to already-classified defect pixels through 8-connectivity neighborhoods. This connectivity criterion ensures that weakly-detected portions of genuine defects are included while isolated noise pixels are excluded. The thresholds were empirically optimized by visual comparison with original microscopy images for a representative subset of frames.

\begin{figure}[h]
    \centering
    \includegraphics[width=1.0\linewidth]{figs/test-framework-2s_40V-S345.png}
    \caption{Images after hysteresis thresholding, binary closing, removing small objects, and skeletonizing. The sample image is a single frame from 40V $10 \times$ without analyzer dynamical data.}
    \label{fig:test_fw_345}
\end{figure}

\subsubsection{Binary closing and removing small objects}

To further refine the segmentation, binary morphological closing (a dilation followed by an erosion) is applied to bridge small gaps and holes within detected string regions, followed by connected component analysis to remove small objects below a chosen length threshold ($16\text{ px}$); together, these steps ensure that only large, continuous defect structures are retained while isolated noise pixels and spurious fragments are effectively suppressed, resulting in cleaner and more accurate representations of the underlying defect network. The closing operation is particularly important for reconnecting defect segments that were artificially separated by noise, while the object removal step ensures that only statistically significant defects contribute to the final density measurement. These morphological operations preserve the topology of connected defect structures while removing noise with sub-pixel resolution.

\subsubsection{Skeletonizing}

 The refined binary image containing identified defect regions is processed through skeletonization, a morphological thinning operation that reduces each detected string to its medial axis; a connected, one-pixel-wide representation of the original structure. The skeletonization algorithm iteratively removes boundary pixels from the binary image while preserving connectivity and topological properties, ensuring that the reduced skeleton maintains the same homotopy as the original object. This compact representation enables precise measurement of string lengths and facilitates reliable analysis of defect network topology, even when original regions are wide or noisy.

To quantify the defect network from processed images, the total string length is estimated by extracting the skeletonized form of each defect and analyzing it with the \textit{Skan} library, which reconstructs the skeleton as a network graph and computes the sum of branch lengths using pixel connectivity and spacing. The cumulative length of all skeleton branches in the field of view directly yields the projected string length $L$. To obtain the string density $\rho$, this total length is divided by the sample volume $V$ (the area imaged multiplied by the cell thickness), resulting in 
\[
\rho = \frac{L}{V}
\]

\subsection{Error analysis}

Uncertainty in the measured defect density arises from multiple sources that must be carefully considered. First, the optical resolution of the microscopy system limits our ability to resolve fine defect structures; defects narrower than approximately 1-2 pixels are likely to be missed or incorrectly classified. This resolution limit is systematic and affects all measurements but becomes more significant at low defect densities where total string lengths are small. Second, the ridge detection and thresholding algorithms introduce variability; repeated processing of the same image with slightly different parameters can yield different results. To quantify this source of error, we applied the analysis pipeline to the same set of representative images (10 frames) with systematically varied hysteresis thresholds ($T_{\text{high}} \in [0.15, 0.25]$, $T_{\text{low}} \in [0.05, 0.12]$) and found density variations of approximately 10-15% depending on parameter choices.

Third, uncertainty arises from the cell thickness measurement ($31.09 \pm 0.44$ μm), which propagates directly into density calculations with relative uncertainty $\approx 1.4\%$. Finally, at very early times when defect density is low, finite-size effects become important; the field of view may not be representative of the bulk system, and defect detection becomes noisier. Given these considerations, we estimate the overall systematic uncertainty in density measurements at approximately $\pm 15-20\%$ across all datasets. Statistical uncertainty from averaging over multiple frames is typically much smaller ($< 5\%$) and is neglected compared to systematic uncertainties.

\section{Defect Dynamics}

\subsection{The Freedericksz Transition}

The Freedericksz transition is a fundamental instability in nematic liquid crystals that occurs when an electric or magnetic field exceeds a critical threshold value. In our experimental geometry with homeotropic alignment (director perpendicular to surfaces) and electric field applied parallel to the cell plates, the transition occurs when the field energy becomes comparable to the anchoring and elastic energies that maintain homeotropic alignment.

The theoretical threshold voltage for the Freedericksz transition in homeotropic geometry can be derived from an energy balance. The distortion free energy density due to director bending is given by $F_d \sim \frac{1}{2}K(\nabla\theta)^2$, where $\theta$ is the director tilt angle from the vertical. The electric field energy density is $F_e \sim -\frac{1}{2}\epsilon_0 \Delta\chi_e E^2 \sin^2\theta$, where $\Delta\chi_e = \chi_e^{\parallel} - \chi_e^{\perp}$ is the dielectric anisotropy and $E$ is the electric field strength. At the threshold, the uniform homeotropic state ($\theta = 0$) becomes unstable, and a spatially modulated configuration begins to form. The critical field strength scales as
\[
E_c \sim \sqrt{\frac{K}{\epsilon_0\Delta\chi_e d^2}}
\]
where $d$ is the cell thickness. This scaling shows that thicker cells have lower threshold voltages (proportional to $d^{-1}$), a behavior confirmed experimentally in our system.

In our experiments, the threshold voltage appears to be approximately 8-10V based on visual observation of the onset of defect formation and birefringence changes at 10V. However, detailed threshold characterization was not the primary focus of this project; rather, we selected voltages well above threshold (10V, 20V, 40V, 60V) to ensure robust defect formation and allow clear observation of subsequent coarsening dynamics. The Freedericksz transition itself is a fascinating phase transition worthy of detailed study, but our project emphasizes the dynamics following transition—specifically, how the resulting defect networks evolve and coarsen over time.

\subsection{Defect formation dynamics}

Following the applied voltage exceeding the Freedericksz threshold, the uniform homeotropic state becomes unstable and the director field begins to develop spatial variations. The precise mechanism by which defects nucleate depends on the specific conditions and the underlying instability. In the context of the Kibble mechanism applied to liquid crystals, as the system is driven through the transition, different regions of the sample undergo the transition at slightly different times and develop different director configurations. Where incompatible orientations meet, topological defects must form to mediate the transition.

Experimentally, we observe the appearance of dark lines in polarized microscopy images beginning within the first few seconds after voltage application. These dark features represent regions where the director has deviated significantly from homeotropic alignment and often indicate the presence of defect lines (disclinations in liquid crystal terminology). The density and spatial distribution of these initial defects depend on the quench rate—in our case, the voltage rise time (essentially instantaneous in our setup, limited by electrical switching times of \(\sim 1\text{ ms}\)). Faster quenches produce higher defect densities, consistent with Kibble predictions.

The initial phase of defect formation (first 1-2 seconds) exhibits rapid changes as the director field explores the available configuration space. Due to the complexity of this transient behavior and the limitations of our analysis pipeline (which performs best on images with moderate defect density), we defer detailed analysis of formation dynamics to future work. Instead, our current study focuses on the more tractable late-time regime (beyond \(\sim 2-3\) seconds) where the defect density has reached a more accessible level and clear scaling behavior emerges.

\subsection{Late-time coarsening dynamics}

\indent Late-time coarsening dynamics were analyzed for 10V, 20V, 40V, and 60V datasets, acquired at $10 \times$ magnification without an analyzer. A linear best-fit was applied to the log-log plots of string length versus time, resulting in the following slopes:
\begin{itemize}
\item 10V: -1.20
\item 20V: -0.80
\item 40V: -0.98
\item 60V: -1.02
\end{itemize}

These slopes are close to the theoretically predicted value of -1, consistent with the expected $\rho(t) \propto 1/t$ scaling in the coarsening regime. The slightly higher error for the 10V and 20V cases may be attributed to a lower density of robust defect lines and increased uncertainty in string length estimation within these datasets. In the 10V case, the defect density remains sufficiently low that finite-size effects and stochastic fluctuations introduce significant noise into individual measurements. The 20V dataset exhibits somewhat better statistics while still showing deviation from the theoretical prediction, possibly indicating that the system has not yet fully entered the asymptotic scaling regime.

The 40V and 60V datasets, by contrast, show excellent agreement with the $\rho(t) \propto 1/t$ scaling law, with slopes very close to the theoretical value of -1. This agreement is remarkable and suggests that the simple one-scale model captures the essential physics of defect coarsening in our system. The slight variation in slopes across different voltages (ranging from -0.80 to -1.20) is consistent with statistical fluctuations expected for a finite defect population in a finite observation volume. The measured slopes can be compared to the theoretical prediction; any systematic deviation from -1 would indicate either: (1) failure of the one-scale model, (2) additional physical processes not captured by the model (such as anisotropic coarsening), or (3) systematic errors in the analysis methodology.

\begin{figure}[h]
    \centering
    \subfloat[Late-time coarsening dynamics for 10V]{\includegraphics[width=0.45\textwidth]{figs/10V-rho_dyn.png}}
    \hspace{0.05\textwidth}
    \subfloat[Late-time coarsening dynamics for 20V]{\includegraphics[width=0.45\textwidth]{figs/20V-rho_dyn.png}}
    \vspace{0.5cm} 
    \subfloat[Late-time coarsening dynamics for 40V]{\includegraphics[width=0.45\textwidth]{figs/40V-rho_dyn.png}}
    \hspace{0.05\textwidth} 
    \subfloat[Late-time coarsening dynamics for 60V]{\includegraphics[width=0.45\textwidth]{figs/60V-rho_dyn.png}}
    \caption{Late-time coarsening dynamics for 10V, 20V, 40V, 60V with $10 \times$ without analyzer datasets. Each plot shows defect density on a logarithmic scale versus time, with the slope of the best-fit line indicating the temporal scaling exponent. Log-log analysis reveals that the data are consistent with $\rho(t) \propto t^{-1}$ scaling, as predicted by the one-scale model of defect network evolution.}
    \label{fig:dyn_string_density}
\end{figure}

The physical mechanism driving the $1/t$ coarsening can be understood from energy and momentum conservation arguments. When the system is dominated by viscous friction (as is typically the case in liquid crystals at room temperature), the driving force for defect motion comes from line tension—the energy per unit length of a defect. As two defects of opposite sign approach each other, they experience an attractive force proportional to their tension and inversely proportional to their separation. The resulting velocity is limited by viscous drag, giving $v \sim T/(\Gamma\xi)$, where $T$ is the line tension, $\Gamma$ is the friction coefficient, and $\xi$ is the typical separation. The energy dissipation rate per unit volume is then $\dot{W} \sim T^2\rho^2/\Gamma$. Energy conservation requires that this dissipation rate equal the rate of change of elastic energy density: $\dot{W} \sim T\dot{\rho}$. Equating these expressions and using $\rho \sim 1/\xi$ (so $\dot{\rho} \sim -\dot{\xi}/\xi^2$) yields the scaling $\rho(t) \propto 1/t$. The universal nature of this scaling—its independence of material parameters like line tension and friction coefficient—explains why we observe similar behavior across different voltage conditions.

\vspace{1cm}
\section{Conclusion}

This semester-long project has successfully demonstrated the value of nematic liquid crystals as a laboratory system for studying cosmologically relevant defect dynamics. Through a combination of careful experimental design, meticulous sample preparation, and advanced computational image analysis, we have characterized defect formation and evolution in a well-controlled system. Our primary findings are:

\begin{enumerate}
\item \textbf{Successful establishment of homeotropic alignment:} The cell preparation procedures yielded high-quality samples with uniform homeotropic alignment, verified through polarized microscopy. The measured cell thickness of $31.09 \pm 0.44$ μm provides a well-characterized geometry for defect studies.

\item \textbf{Development of quantitative analysis methods:} The multi-stage image processing pipeline, combining ridge detection, morphological operations, and skeletonization, successfully extracts quantitative measures of defect density from optical microscopy data. While systematic uncertainties of 15-20% remain, the method provides reproducible results suitable for testing theoretical predictions.

\item \textbf{Observation of Freedericksz transition and defect formation:} Clear visual evidence of defect formation is observed above approximately 8-10V, consistent with the expected Freedericksz threshold. Defect nucleation occurs within the first few seconds following voltage application, with subsequent evolution governed by network coarsening.

\item \textbf{Quantitative validation of theoretical predictions:} Late-time coarsening dynamics (beyond 2-3 seconds) exhibit scaling behavior $\rho(t) \propto t^{-1}$ with exponents agreeing with the one-scale model within experimental uncertainty. The 40V and 60V datasets show particularly good agreement (slopes of -0.98 and -1.02, compared to theoretical prediction of -1.00), providing experimental confirmation of fundamental concepts in defect physics.

\item \textbf{Confirmation of universal coarsening laws:} The observation of similar scaling exponents across different voltage regimes (despite variations in defect densities and initial formation conditions) demonstrates the universal nature of viscosity-dominated defect coarsening.
\end{enumerate}

\subsection*{Works need to be done}

\begin{itemize}
    \item \textbf{Defect formation dynamics:} Detailed analysis of the initial formation phase (0-2 seconds) requires improved image processing methods capable of handling high defect densities and complex topologies. Developing algorithms for automated defect tracking and classification would enable quantitative comparison with Kibble mechanism predictions.
    
    \item \textbf{String intercommutation events:} Documentation and statistical analysis of intercommutation events (where crossing defects exchange partners) would provide information about defect-defect interactions. Such events are rare but observable in our videos; quantifying their frequency and mechanism would validate predictions from defect dynamics simulations.
    
    \item \textbf{Analysis of additional datasets:} Expansion to include datasets from different magnifications (5×, 20×), different optical configurations (crossed polarizers, alternative analyzer angles), and different measurement times would improve statistical robustness. Analysis of the early-time regime would enable direct testing of Kibble predictions.
    
    \item \textbf{Defect dynamics under magnetic field quench:} Application of magnetic fields rather than electric fields would provide a complementary probe of defect formation. Because the magnetic interaction does not involve ionic currents, it offers advantages over electric field control. Comparing scaling behavior under magnetic versus electric quenches would further test the universality of defect coarsening.
    
    \item \textbf{Three-dimensional reconstruction:} While our current analysis focuses on projected densities in 2D images, the actual defects are three-dimensional strings extending through the sample thickness. Confocal microscopy or other volumetric techniques could enable full 3D characterization of defect networks.
    
    \item \textbf{Comparison with simulations:} Numerical simulations of nematic liquid crystal dynamics under similar conditions would enable detailed comparison between experimental observations and theoretical predictions. Such simulations could elucidate mechanisms of defect formation and annihilation beyond what can be inferred from 2D projections alone.
\end{itemize}

% Bibliography
\bibliographystyle{naturemag}
\bibliography{references}

% Appendices (if needed)
\appendix
\section{Supplementary Materials}

This appendix contains additional details regarding experimental procedures, data analysis parameters, and theoretical background.

\subsection{Cell preparation troubleshooting}

Several technical challenges may arise during cell preparation. Incomplete ITO etching can result from inadequate acid concentration or insufficient reaction time; these should be verified through multimeter testing showing complete loss of conductivity in patterned regions. Spacer bead aggregation during cell assembly can be prevented by careful manipulation and ensuring adequate spacing before epoxy application. Air bubbles trapped during liquid filling can sometimes be removed through gentle heating above the liquid crystal clearing point while monitoring the cell under the microscope. Surface inhomogeneities can develop if the DMOAP surfactant is not applied uniformly; multiple coatings or extended equilibration times may be necessary.

\subsection{Temperature control and stability}

Maintaining constant temperature (27.5°C) throughout experiments is critical because the elastic constants and viscosity of nematic liquid crystals are strongly temperature-dependent. Fluctuations of ±1°C can introduce variations in the Freedericksz threshold of several volts. The temperature controller should be verified against an independent thermometer and given adequate time (typically 30+ minutes) to reach thermal equilibrium after initial setup.

\subsection{Voltage application and waveform characterization}

The AC waveform characteristics warrant careful attention. Frequency must be above the ionic relaxation frequency (typically 100-500 Hz for MBBA) to minimize ionic current contributions. Waveform distortion from the amplifier can introduce harmonics that affect threshold behavior. The RMS voltage reported by the function generator should be independently verified using a precision voltmeter, as differences between intended and actual voltages can lead to misinterpretation of threshold measurements.

\section{Acknowledgements}

I gratefully acknowledge the guidance and support of Prof. Pramoda Kumar and Ankit Gupta throughout this project. Helpful discussions with colleagues in the physics department, particularly regarding liquid crystal physics and image processing techniques, significantly improved the quality of this work. I thank the Ashoka University Department of Physics for providing laboratory facilities and equipment. Finally, I acknowledge the open-source software communities that developed scikit-image, skan, and other tools used in this analysis.

% Add any supplementary materials, equations, or detailed procedures here

\end{document}