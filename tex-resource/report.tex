\documentclass[12pt]{article}
\usepackage[utf8]{inputenc}
\usepackage{fancyhdr}

\usepackage[T1]{fontenc}
\usepackage{amsmath,amsfonts,amssymb}
\usepackage{indentfirst, geometry, hyperref, fontspec, enumitem, graphicx, subfig, gensymb, abstract, wrapfig}
\usepackage{cite}

\usepackage[dvipsnames]{xcolor}
\usepackage{textcomp}


\setmainfont{Georgia}


% Page setup
\pagestyle{fancy}
\fancyhf{}
\setlength{\headwidth}{1.12\textwidth}
\fancyhead[R]{\thepage}
\fancyhead[L]{\leftmark}
\renewcommand{\headrulewidth}{1pt}

\geometry{
    a4paper,
    left=2.0cm,
    right=2.0cm,
    top=3.0cm,
    bottom=2.0cm
}

% Hyperref setup
\hypersetup{
    colorlinks=true,
    linkcolor=RoyalBlue,
    citecolor=Green,
    filecolor=magenta,      
    urlcolor=RoyalBlue,
    pdftitle={Laboratory Studies of Cosmology-Inspired Defect Dynamics in Liquid Crystals},
    pdfauthor={Anik Mandal},
    bookmarksopen=true,
    bookmarksnumbered=true
}

% Custom commands
\newcommand{\email}[1]{\texttt{#1}}
\newcommand{\university}{Ashoka University}
\newcommand{\location}{Sonipat, India}


\begin{document}

% Title page
\begin{titlepage}
    \centering
    \vspace*{2cm}
    {\huge\bfseries Laboratory Studies of Cosmology Inspired Defect Dynamics in \\Liquid Crystals\par}
    \vspace{2cm}
    {\Large Semester Long Project Report\par}
    \vspace{1.5cm}
    {\large
    \textbf{Anik Mandal}$^{1,*}$ \\[0.5em]
    \begin{tabular}{r l}
        SLP Guide &:  Prof. Pramoda Kumar$^{1}$ \\
        SLP Mentor &: Ankit Gupta$^{1}$ \\
        PhD Supervisor &:  Prof. Somak Raychaudhury$^{1}$ \\[1em]
    \end{tabular}

  
    $^1$\university, \location \\
    $^{*}$Email: \href{mailto:anik.mandal_phd24@ashoka.edu.in}{anik.mandal\_phd24@ashoka.edu.in}
    }\\
    \vspace{2cm}
    \includegraphics[width=0.2\linewidth]{tex-resource/ashoka-logo.png}
    \vfill
    {\large \today\par}
\end{titlepage}

% Abstract
\begin{abstract}
Spontaneous symmetry breaking in the early universe is expected to generate topological defects whose dynamics 
encode high-energy physics beyond the reach of terrestrial accelerators, a phenomenon modeled here using 
nematic liquid crystals to test the Kibble–Zurek mechanism (KZM) and one-scale coarsening models. In this 
work, a homeotropically aligned MBBA cell was driven across the Fréedericksz transition using controlled AC 
electric fields, while high-speed optical microscopy recorded defect formation and evolution. A dedicated 
computational framework was developed to extract defect line density as a function of time and control 
parameters. For late-time dynamics in a cell with pre-existing defects, the defect density exhibits a power-law 
decay $\rho(t) \propto t^{-1}$ within uncertainties, consistent with string-dominated one-scale coarsening, 
whereas a newly prepared, defect-free cell initially forms closed-loop structures with substantially slower 
effective exponents, reflecting a crossover from loop-dominated to line-dominated regimes. Attempts to probe KZ 
scaling in defect formation using high-frequency (10 kHz) sawtooth field ramps revealed a freeze-out of 
director dynamics and strong stochasticity, while lowering the frequency led to electroconvection; these 
results delineate the practical parameter space for cosmology-inspired experiments in nematic liquid crystals, 
highlighting the robustness of late-time coarsening scaling while identifying specific challenges in accessing 
defect-formation scaling in electrically driven MBBA cells.
\end{abstract}

% Table of Contents
\setcounter{tocdepth}{2}
\tableofcontents
\newpage

% Main content
\section{Introduction}
Spontaneous symmetry breaking is a fundamental phenomenon where the lowest energy state (ground state) of a 
system exhibits less symmetry than the physical laws governing it. At high temperatures or energy levels, the 
system exists in a symmetric, disordered state. However, as the system cools below a critical threshold, it 
becomes energetically favorable to transition to an ordered state. To do so, the system must spontaneously 
select a specific orientation from a set of equally valid possibilities.

A classic analogy is a ball sitting at the peak of a ``Mexican hat" potential: while the shape of the potential 
itself is perfectly symmetric, the ball must eventually roll down into the valley to minimize its energy. In 
doing so, it chooses a random direction, thereby breaking the rotational symmetry. In the context of both 
cosmology and condensed matter physics, this choice happens independently in different regions of space. When 
these disparate regions eventually meet, their orientations often fail to align smoothly, leading to the 
formation of stable discontinuities known as topological defects\cite{Kibble1976TopologyStrings}.

\subsection{Symmetry-breaking in early Universe}
\begin{wrapfigure}[24]{r}{0.4\textwidth}
    \centering
    \includegraphics[width=0.95\linewidth]{figs/imported/brief_histUniv.png} 
    \caption{Brief history of the universe: different symmetry-breaking epochs at early universe. (\href{https://www.ctc.cam.ac.uk/outreach/origins/big_bang_three.php}{Source})}
    \label{fig:brief_histUniv}
\end{wrapfigure}

Current cosmological models indicate that during the Planck epoch and the earliest high-energy 
stages of cosmic evolution, all fundamental interactions were unified into a single force. At these extreme 
energy scales, the Universe existed in a maximally symmetric state in which the distinctions between the 
strong, weak, and electromagnetic interactions were absent. As the universe expanded and cooled below a series 
of critical temperatures, it underwent successive phase transitions driven by spontaneous symmetry breaking 
\cite{CentreTransitions, EarlyAl.}. These transitions are not merely abstract mathematical constructs; rather, they 
play a fundamental role in reshaping the vacuum state of the universe. In this sense, they are closely analogous to 
phase transitions in condensed matter systems, such as the freezing of water into ice or the emergence of 
ferromagnetic order.

In particle physics, the unification of weak and electromagnetic interactions into an electroweak gauge theory 
based on $SU(2)\times U(1)$ is well established. Together with Quantum Chromodynamics (QCD), the $SU(3)$ theory 
of strong interactions, this naturally motivates Grand Unified Theories (GUTs), in which the Standard Model 
forces are embedded within a larger simple gauge group, $G$. This unification is possible because the strengths 
of the fundamental forces are not fixed but run with the energy scale. At higher energies, such as those present in 
the early Universe, these couplings (or, the strength of the forces) become more similar, allowing the different 
forces to behave as a single unified interaction at a grand unification scale of order $10^{15}$ GeV ($\sim 
10^{-4}$ of the Planck mass) \cite{Kibble1982PhaseUniverse}. 

The subsequent evolution of the Universe therefore involved a hierarchy of symmetry-breaking events. A 
convenient and physically transparent way to describe symmetry breaking during these cosmological phase 
transitions is through a Higgs-type scalar field $\phi$. For illustrative purposes, consider a simple theory 
with symmetry group $G=U(1)$ and a complex scalar field governed by the self-interaction potential
\cite{Vilenkin1985CosmicWalls},

\begin{equation}
V(\phi)=\frac{\lambda}{4}\left(\phi^\dagger\phi-\eta^2\right)^2 .
\end{equation}

At zero temperature, the minima of this potential occur at nonzero field values $|\phi|=\eta$, indicating 
spontaneous symmetry breaking. The field acquires a vacuum expectation value $\langle\phi\rangle=\eta 
e^{i\theta}$, where the magnitude is fixed but the phase $\theta$ is arbitrary. Consequently, the set of 
degenerate vacuum states forms a continuous vacuum manifold, which in this case corresponds to a circle in the 
complex $\phi$ plane.

In the hot early Universe, however, thermal effects play a crucial role. At finite temperature, the effective 
potential for $\phi$ receives additional contributions from interactions with the thermal bath. In the 
high-temperature limit, this effective potential can be written as,

\begin{equation} \label{eq:eff_potential_T}
V_T(\phi)=A T^2\phi^\dagger\phi + V(\phi),
\end{equation}

which introduces a temperature-dependent mass term (quadratic term of equation-\ref{eq:eff_potential_T}) 
$m^2(T)=A T^2-\lambda\eta^2$. Here the dimensionless constant $A$ is a combination of the self-coupling 
$\lambda$ and other couplings of the field $\phi$ (e.g., Yukawa couplings and gauge coupling). For temperatures 
above the critical value $T_c=\eta\sqrt{\lambda/A}$, the mass squared is positive and the effective potential 
is minimized at $|\langle\phi (T>T_c)\rangle|=0$, corresponding to a symmetric phase. As the Universe cools and the 
temperature drops below $T_c$, the mass squared becomes negative, rendering the symmetric state unstable and 
driving the field toward a new minimum with a non-zero expectation value given by,

\begin{equation}
|\langle\phi(T<T_c)\rangle|^2=\left( \eta^2-\frac{A}{\lambda}T^2\right) =\eta^2 \left(1 -\frac{T^2}{T_c^2}\right) \  \{A>0\}
\end{equation}

This marks the onset of spontaneous symmetry breaking and a qualitative change in the vacuum structure.

\begin{wrapfigure}[16]{r}{0.4\textwidth}
    \centering
    \includegraphics[width=0.95\linewidth]{figs//imported/higgs_broken.png}
    \caption{Spontaneous symmetry breaking of the Higgs field below the critical temperature, resulting in the generation of massive excitations via the Higgs mechanism. (\href{https://medium.com/@symmetrybreaking/spontanteous-symmetry-breaking-the-mass-creation-mechanism-9a37fb725080}{Source})}
    \label{fig:higgs_broken}
\end{wrapfigure}

The first major transition involved the breakdown of the GUT symmetry to the Standard Model symmetries, 
likely depositing massive topological defects. This was followed much later by electroweak symmetry breaking at 
an energy scale of approximately $100$ GeV \cite{Kibble1982PhaseUniverse}. These transitions represent dramatic 
changes in the vacuum structure of the Universe and underpin several fundamental phenomena, including the 
generation of particle masses through the Higgs mechanism (see figure-\ref{fig:higgs_broken}) and the emergence 
of the observed baryon asymmetry through mechanisms involving CP violation. A fundamental constraint governing 
these transitions is causality; physical effects cannot propagate faster than the speed of light, $c$. 
Consequently, at any given time $t$, regions of the Universe separated by a distance greater than the horizon 
distance $d=ct$ are causally disconnected.

During a symmetry-breaking phase transition, these isolated regions must independently decay into a specific 
minimum energy state within the vacuum, $\mathcal{M}$. Due to the stochastic nature of this process, causally 
separated regions often settle into different vacuum states, much like crystallizing ice forming misaligned 
grains. The boundaries between these misaligned regions manifest as topological defects \cite{CentreDefects}. 
These are stable configurations of energy, formed where the vacuum choices of neighboring regions cannot be 
smoothly reconciled. 

To understand the variety of defects that can emerge, one must look at the specific geometry of the potential 
landscape. In a simplified model with two distinct minima (e.g., positive and negative states), a domain wall 
forms at the interface. In theories where the vacuum manifold has a complex topology, specifically containing 
holes or non-contractible loops, the underlying field can wrap around these topological features. This wrapping 
creates a mismatch in the field's orientation, resulting in stable, linear defects of trapped high energy known 
as cosmic strings \cite{Vilenkin1985CosmicWalls}. The specific nature of these defects, whether they manifest 
as surfaces (domain walls), lines (cosmic strings), or points (monopoles), depends strictly on the topology of 
the broken symmetry group.

Mathematically, this dependency is classified using homotopy groups, $\pi_n(\mathcal{M})$, which describe the 
connectivity of the vacuum manifold defined as the quotient space, $\mathcal{M} = G/H$. Here, $G$ represents 
the full, underlying symmetry group of the high-energy theory and $H$ denotes the subgroup of symmetries that 
remain unbroken in the vacuum state. For instance, a disconnected manifold ($\pi_0(\mathcal{M}) \neq 
\mathbb{I}$) yields domain walls, a manifold with non-contractible loops ($\pi_1(\mathcal{M}) \neq \mathbb{I}$) 
produces cosmic strings, and non-contractible 2-spheres ($\pi_2(\mathcal{M}) \neq \mathbb{I}$) result 
monopoles. Furthermore, if the manifold supports non-trivial mappings of the 3-sphere ($\pi_3(\mathcal{M}) \neq 
\mathbb{I}$), the resulting defects are known as textures, which manifest as unstable, twisted field 
configurations rather than singular boundaries.

\begin{figure}[h]
    \centering
    \subfloat[The Kibble mechanism for the formation of domain walls. (Source-\cite{CentreDefects})]{\includegraphics[width=0.45\textwidth]{figs/imported/domain_wall.png}}
    \hspace{0.05\textwidth}
    \subfloat[The Kibble mechanism for the formation of cosmic strings. (Source-\cite{CentreDefects})]{\includegraphics[width=0.45\textwidth]{figs/imported/string_defect.png}}
    \label{fig:kibble_defect_form}
\end{figure}

The cosmological implications of topological defects are strictly dependent on the specific symmetry 
broken. Defects such as domain walls and magnetic monopoles possess such massive energy densities that 
their existence would dominate the universal energy budget, leading to evolution scenarios that 
contradict observational data \cite{CentreDefects}; models predicting these are therefore largely ruled out. 
Conversely, cosmic strings are cosmologically viable. They may have served as gravitational ``seeds" for 
large-scale structure formation \cite{Vilenkin1985CosmicWalls, Shellard1988ClustersStrings}, contributed to 
anisotropies in the Cosmic Microwave Background (CMB), and could potentially account for a portion of the 
universe's dark matter.

Consequently, topological defects provide a unique observational window 
\cite{Kibble2007Phase-transitionUniverse} into the physics of the extremely early Universe, an energy 
scale inaccessible to terrestrial particle accelerators. However, to rigorously evaluate these scenarios, 
a precise understanding of defect dynamics is required. Before current observations can be used to 
constrain high-energy theory, it is imperative to develop a robust model of how cosmological defects evolve and 
interact throughout cosmic history.


\subsection{Kibble-Zurek mechanism}

The Kibble-Zurek mechanism (KZM) is a theoretical framework that unifies the non-equilibrium dynamics of symmetry-
breaking phase transitions across vastly different energy scales. Originally proposed by T.W.B. Kibble to explain 
the formation of cosmological defects in the early universe\cite{Kibble1976TopologyStrings}, the theory was later 
extended by W.H. Zurek to condensed matter systems\cite{Zurek1985CosmologicalHelium, Zurek1996CosmologicalSystems}. 
By relying on the universality of critical dynamics, the KZM links cosmological topological defects to accessible 
laboratory analogues like vortices in superfluids or disclinations in liquid 
crystals\cite{Chuang1991CosmologyCrystals, Chuang1991Late-timeCrystal, Yurke1992CoarseningCrystals, Shen2020AnnihilationCrystals, Blanc2005DynamicsBackflow, Dierking2025MachineNanoparticles}.

\vspace{0.5cm}

\noindent{\textbf{The Nematic Order Parameter:}} To apply this universal framework to the specific case of liquid 
crystals, we must first define the physical system. Liquid crystals \cite{Gennes1993TheCrystals, Palffy-Muhoray2007TheCrystals, Chuang1991CosmologyCrystals} constitute a class of organic systems that exhibit distinct 
mesophases intermediate between isotropic liquids and crystalline solids, characterized by long-range orientational 
order described by a director field $\mathbf{n}(\mathbf{r})$.

The key feature of this order parameter is its symmetry. In the high-temperature isotropic phase, the molecules are 
randomly oriented, possessing full rotational symmetry ($O(3)$). As the system cools into the nematic phase, this 
symmetry breaks, and the molecules align along a common axis. However, because the molecules are non-polar (head-to-
tail symmetric), the states $\mathbf{n}$ and $-\mathbf{n}$ are physically indistinguishable. This specific symmetry 
breaking ($O(3) \to O(2)$) creates a complex ``vacuum manifold'' that allows for the formation of stable topological 
defects, specifically the $\pm 1/2$ disclination strings.

Any distortion in this alignment costs energy, quantified by the Frank elastic free energy, $F$. In its general 
form, the free energy density is given by ,
\begin{equation}
F = \frac{1}{2} \left\{ K_1 (\nabla \cdot \mathbf{n})^2 + K_2 (\mathbf{n} \cdot \nabla \times \mathbf{n})^2 + K_3 |\mathbf{n} \times \nabla \times \mathbf{n}|^2 \right\}
\end{equation}
where $K_1$, $K_2$, and $K_3$ are the splay, twist, and bend elastic constants.

Because liquid crystals are highly viscous, inertial effects are negligible. The director field therefore evolves 
purely to minimize the free energy, balancing viscous drag against elastic forces. The general equation of motion 
is\cite{Yurke1992CoarseningCrystals},

\begin{equation}
\gamma \frac{\partial n_{\alpha}}{\partial t} = - \frac{\delta F}{\delta n_{\alpha}}
\end{equation}
where $\gamma$ is the rotational viscosity and the derivative includes the constraint $|\mathbf{n}|^2=1$.

By applying the ``one-constant approximation'' ($K_1=K_2=K_3=K$), this simplifies to the \textit{damped nonlinear 
sigma model} which is similar to the nonlinear sigma model used by cosmologists to study the evolution of a universe
populated by global defects \cite{Yurke1992CoarseningCrystals}:
\begin{equation}
\gamma \frac{\partial n_{\alpha}}{\partial t} = K \left[ \nabla^2 n_{\alpha} + (\nabla n_{\beta}) \cdot (\nabla n_{\beta}) n_{\alpha} \right]
\end{equation}

This diffusive equation governs the relaxation process that drives the late-time coarsening of the defect network.

\vspace{0.5cm}

\subsubsection{Defect formation dynamics} With the system defined, we can now predict how defects form during the 
transition. The core quantitative prediction of the KZM describes how the density of resulting defects scales with 
the speed of the transition. Consider a continuous phase transition driven by a control parameter $\epsilon(t)$, 
such as reduced temperature $|T-T_c|/T_c$, pressure, or electric field, that varies linearly with time $t$ across 
the critical point at $t=0$:

\begin{equation}
\epsilon(t) = \frac{|t|}{\tau_Q}
\end{equation}

Here, $\tau_Q$ represents the characteristic quench timescale; a smaller $\tau_Q$ implies a faster transition.

As the system approaches the critical point, its internal dynamics slow down significantly. According to Landau-
Ginzburg theory, the equilibrium relaxation time $\tau(t)$ and the correlation length $\xi(t)$ diverge as functions 
of the control parameter,

\begin{align}
\tau(t) &\propto |\epsilon(t)|^{-\nu z} \\
\xi(t) &\propto |\epsilon(t)|^{-\nu}
\end{align}

where $\nu$ is the correlation-length critical exponent and $z$ is the dynamical critical exponent.

The system can maintain equilibrium only as long as its relaxation time $\tau(t)$ is faster than the time scale on 
which the environment is changing, represented by the time remaining until the transition $|t|$. As $\tau(t)$ 
diverges near the critical point, it inevitably overtakes $|t|$. At this moment, the system's reaction time becomes 
too slow to adapt to the changing conditions, and the domain structure effectively ``freezes''. This ``freeze-out'' 
occurs at a characteristic time $\hat{t}$ when the two timescales become comparable,

\begin{equation}
\tau(\hat{t}) \approx \hat{t}
\end{equation}

By substituting the power-law dependence of the relaxation time into this condition, we can solve for the freeze-out 
time $\hat{t}$ in terms of the quench rate,

\begin{equation}
\tau(\hat{t})=\left( \frac{\hat{t}}{\tau_Q} \right)^{-\nu z} \propto \hat{t} \implies \hat{t} \propto  \tau_Q^{(\frac{\nu z}{1+\nu z})}
\end{equation}

Consequently, the control parameter at freeze-out scales as,
\begin{equation}
\hat{\epsilon} = \epsilon(\hat{t}) \propto \tau_Q^{-(\frac{1}{1+\nu z})}
\end{equation}

The density of the resulting topological defects is determined by the configuration of the system at the moment of 
freeze-out. The characteristic size of the ordered domains is set by the correlation length $\hat{\xi}$ at time 
$\hat{t}$,

\begin{equation}
\hat{\xi} = \xi(\hat{t}) \propto |\hat{\epsilon}|^{-\nu} \propto \left(\tau_Q^{-(\frac{1}{1+\nu z})}\right)^{-\nu} \propto \tau_Q^{(\frac{\nu}{1+\nu z})}
\end{equation}

Assuming that defects form at the boundaries of these domains, the initial defect density $\rho$ scales inversely 
with the volume (or area) of the domains. If $d$ is the effective dimension of the defect (where $d = 
D_{\text{space}} - D_{\text{defect}}$), then $\rho \propto \hat{\xi}^{-d}$. This yields the Kibble-Zurek scaling law 
for defect formation,

\begin{equation}
    \boxed{\rho \propto \tau_Q^{-\frac{d\nu}{1+\nu z}}}
\end{equation}

This equation predicts a universal power-law dependence on the quench rate. For example, in a mean-field transition 
(where $\nu=1/2$ and $z=2$) producing line defects in three dimensions ($d=2$), the exponent simplifies to the 
standard result often cited in literature\cite{Fowler2017KibbleZurekCrystal, Kibble2007Phase-transitionUniverse},

\begin{equation}
    \rho \propto \tau_Q^{-1/2}
\end{equation}

This scaling law has been experimentally verified in diverse systems, providing strong evidence for the universality 
of the mechanism.

\subsubsection{Late-time coarsening dynamics} 
Following their formation, the defect network evolves to minimize the system's Frank elastic free energy. This 
process is effectively described by the ``one-scale" model, which simplifies the complex network by assuming it is 
defined by just one characteristic length, $\xi(t)$, that changes with time. This scale $\xi(t)$ simultaneously 
represents the typical radius of curvature of the defects and the average separation distance between them.

In condensed matter systems, such as nematic liquid crystals, the motion of defects is typically overdamped, meaning 
that frictional forces dominate over inertial effects. Strictly speaking, the string line tension is proportional to 
$\ln(R/R_c)$, where $R$ is the radius of the disclination line (essentially the distance to the nearest neighbor, $R 
\approx \xi$) and $R_c$ is the core radius\cite{Yurke1992CoarseningCrystals}. Similarly, the viscous dissipation is 
proportional to $\ln(R/R_c)$. However, because these logarithmic dependences are weak functions of $R$, to a good 
approximation one can model the dynamics of a string as having a constant line tension $T$ and a constant mobility 
related to the friction coefficient per unit length $\Gamma$.

Under this approximation, the dynamics are governed by the balance between two competing forces per unit length. The 
tension force $F_T$, which acts to straighten curved defect segments, scales inversely with the local curvature 
radius $\xi$,

\begin{equation}
F_T \approx \frac{T}{\xi}
\end{equation}

Opposing this motion is the frictional drag force $F_f$, arising from the fluid 
viscosity, which is proportional to the defect velocity $v$,

\begin{equation}
F_f \approx \Gamma v
\end{equation}

By equating these forces ($F_T \approx F_f$), we 
determine the characteristic terminal velocity \cite{Chuang1991CosmologyCrystals} of the defect network,

\begin{equation}
v \approx \frac{T}{\Gamma \xi}
\end{equation}

As the network coarsens, energy is dissipated through friction. The rate of energy loss per unit volume, $dW/dt$, 
corresponds to the work done against the frictional force. For a defect network with density $\rho \approx \xi^{-d}$ 
(where $d$ is an effective dimension scaling exponent), the power dissipated is,

\begin{equation}
\frac{dW}{dt} \approx - (F_f v) \rho \approx - (\Gamma v^2) \frac{1}{\xi^d}
\end{equation}

Substituting the expression for the characteristic velocity we have,

\begin{equation}
\frac{dW}{dt} \approx - \Gamma \left( \frac{T}{\Gamma \xi} \right)^2 \frac{1}{\xi^d} = - \frac{T^2}{\Gamma \xi^{d+2}}
\end{equation}

The total energy density of the network is given by $W \approx T \rho$. Consequently, the rate of energy loss is 
directly proportional to the rate of change in defect density,

\begin{equation}
\frac{dW}{dt} = T \frac{d\rho}{dt}
\end{equation}

By equating the two expressions for energy dissipation and expressing the length scale in terms of density (noting 
that $\xi^{-(d+2)} = (\rho^{1/d})^{d+2} = \rho^{\frac{d+2}{d}}$), we obtain the evolution equation,

\begin{equation}
T \frac{d\rho}{dt} \approx - \frac{T^2}{\Gamma} \rho^{\frac{d+2}{d}} \implies \frac{d\rho}{dt} \propto -\rho^{\frac{d+2}{d}}
\end{equation}

For the specific case of line defects (strings) in three dimensions, the defect density is defined as length per 
unit volume, which corresponds to $d=2$ (since $\rho \propto \xi^{-2}$). Substituting $d=2$ into the evolution 
equation yields $d\rho/dt \propto -\rho^2$. Integrating this differential equation leads to the universal scaling 
law for late-time dynamics,

\begin{equation}
\boxed{\rho(t) \propto t^{-1}}
\end{equation}

For domain walls ($d=1$) in two or, three dimensions this becomes,
\begin{equation}
\boxed{\rho(t) \propto t^{-\frac{1}{2}}}
\end{equation}

\subsection{Project objectives}

This work establishes nematic liquid crystals \cite{Gennes1993TheCrystals} as a robust laboratory platform for 
investigating the dynamics of topological line defects (strings), providing a direct analogue to those 
predicted in early-universe cosmology. The primary objective is to experimentally validate the Kibble-Zurek 
mechanism (KZM), which governs defect production during symmetry-breaking phase transitions, and the one-scale 
scaling model.

We specifically test quantitative KZM predictions by performing electric-field quenches at varying rates 
$\tau_{Q}$, aiming to confirm the defect density scaling $\rho(0)\propto\tau_{Q}^{-1/2}$. High-speed 
videography is employed to capture the full temporal evolution of the defect network, from initial formation 
through late-time coarsening. The experimental methodology is detailed in Section \ref{sec:Exp}, followed by 
the computational image analysis framework in Section \ref{sec:Comp}. Finally, we present our quantitative 
outcomes in Section \ref{sec:Result} and conclude with a discussion of implications in Section \ref{sec:Conc}.


\newpage
\section{Experimental Setup}\label{sec:Exp}
We employ liquid crystals for this study because they occupy a unique experimental 
advantage. Unlike ordinary fluids (e.g., water), which lack internal directional structure and thus cannot 
support topological defects in the fluid phase, liquid crystals possess the continuous symmetry breaking 
essential for simulating cosmic string formation.  Furthermore, unlike quantum fluids (e.g., superfluid 
helium), which require extreme cryogenic environments, liquid crystals exhibit defect dynamics on accessible 
macroscopic timescales under ambient conditions.

The experiment utilizes a homeotropic alignment, where molecules are anchored perpendicular to the substrate. 
In this uniform state, the system appears optically isotropic. However, during rapid phase transitions 
(quenches), topological defects form where the local order is disrupted. Due to the material's birefringence, 
the rapidly twisting director field near these defects creates sharp refractive index gradients. These 
gradients act as cylindrical lenses, refracting incident light away from the microscope's collection aperture, 
manifesting as distinct dark strings against a bright background. This high-contrast visualization enables 
precise tracking of the defect network. Because the topological constraints of liquid crystals map directly 
onto high-energy field theories, this system serves as a rigorous laboratory analogue for verifying the Kibble-
Zurek mechanism and the non-equilibrium dynamics of cosmic strings.

\subsection{Preparing empty cell}
Preparation of liquid crystal cells follows a series of well-established, standardized steps. The standard 
workflow is outlined below.

\subsubsection{Patterning:}
ITO-coated glass slides are cut and patterned to create defined electrode regions. The glass is placed on a 
cutting mat and the non-coated side is cut to 4 cm × 2 cm dimensions. A multimeter verifies the coated side. 
The coated surface is then marked with 1 cm wide cello tape, leaving 0.5 cm gaps at both ends. This protects 
the desired conducting area.

\subsubsection{Etching:}
In a fume hood, concentrated HCl and zinc powder are added to a beaker containing the patterned glass (coated 
side up). The mixture etches the exposed ITO for 30 minutes, removing it from unprotected areas while leaving 
the tape-covered region intact. After etching, the glass is carefully removed with tweezers and rinsed 
thoroughly with distilled water and a dilute base to neutralize all acid residue. The glass is dried and 
transferred to the cleaning stage.

\subsubsection{Post etching treatment:}
 The etched glass is cut vertically down the middle to produce two symmetric slabs. Both slabs are scrubbed 
 with soap and isopropyl alcohol (IPA) to remove loose particles. They are then placed in a petri dish with a 
 soap-water mixture and sonicated for 15 minutes at 40–50°C. After sonication, the slabs are rinsed with fresh 
 IPA and dried with nitrogen gas, ensuring no residue remains.
 
\begin{wrapfigure}[9]{r}{0.4\textwidth}
    \centering
    \includegraphics[width=0.95\linewidth]{figs/imported/DMOAP.png} 
    \caption{DMOAP structure\hfill\\ ($C_{26}H_{58}CLNO_3Si$) (\href{https://pubchem.ncbi.nlm.nih.gov/compound/Quat-silsesquioxane}{Source})}
    \label{fig:DMOAP}
\end{wrapfigure}

\subsubsection{Applying surfactant (DMOAP):}
A 0.2\% DMOAP surfactant solution is prepared by mixing 160 $\mu$L of DMOAP with 80 mL of distilled water. The 
glass slabs are held vertically and dipped into this solution for 5 minutes, allowing the methoxy groups of the 
surfactant to hydrolyze and form hydrogen bonds with the hydroxyl groups naturally present on the glass 
surface. The slabs are then rinsed with distilled water, dried with nitrogen gas, returned to the petri dish, 
and baked in a hot air oven at 110$^\circ$C for 60 minutes. This thermal treatment drives a condensation 
reaction that converts the initial hydrogen bonds into robust, covalent siloxane (Si-O-Si) linkages.  These 
bonds rigidly anchor the long alkyl chains perpendicular to the substrate, creating a steric barrier that 
induces the desired homeotropic alignment of the liquid crystal molecules.

\subsubsection{Cell assembly:}
Both treated slabs are assembled as a sandwich with spacer beads (23 μm diameter) placed 1 cm apart around the 
edges. The spacers create a uniform gap between the glass plates. The assembled cell is pressed moderately for 30 
minutes to stabilize the spacers. The cell edges are then sealed with epoxy and allowed to cure thoroughly for at 
least a day to ensure a strong bond. Once the epoxy is fully set, electrical wires are connected to the ITO 
electrodes by carefully soldering with indium.

\subsection{Capacitance-based cell thickness measurement}
The thickness of the liquid crystal cell was determined using a capacitance-based measurement technique. 
The cell was modeled as a parallel-plate capacitor, for which the capacitance \(C\) is related to the cell 
thickness \(d\) by
\begin{equation}
    d = \frac{\varepsilon_0 \varepsilon_r A}{C},
\end{equation}
where \(\varepsilon_0\) is the vacuum permittivity, \(\varepsilon_r\) is the relative permittivity (taken 
to be \(\approx 1\) for air), and \(A\) denotes the effective electrode area.

Capacitance measurements were performed using a precision LCR meter. 
To accurately estimate the effective area, only the region of overlap between the patterned ITO electrodes 
was considered, while edge regions and spacer areas were excluded to minimize systematic errors.

For an effective electrode area of \((8.0 \pm 0.4)\times 10^{-5}\,\mathrm{m}^2\), the measured capacitance 
was \(C = 16.23\,\mathrm{pF}\). 
Substituting these values into above equation yields a cell thickness of
\begin{equation}
    d = 43.62 \pm 2.18\,\mu\mathrm{m}.
\end{equation}

\newpage
\begin{wrapfigure}[8]{r}{0.4\textwidth}
    \centering
    \includegraphics[width=0.95\linewidth]{figs/imported/MBBA.png}
    \caption{MBBA (\href{https://en.wikipedia.org/wiki/MBBA}{Source})}
    \label{fig:placeholder}
\end{wrapfigure}

\subsection{Preparing homeotropic cell}
The assembled LC cell is placed on a temperature control device programmed with a three-stage thermal protocol: 
\begin{enumerate}
    \item heat to $50\degree C$ at a rate $20\degree C/min$ and hold for 10 mins,
    \item cool to $36\degree C$ at a rate $1\degree C/min$ and hold for 5 mins,
    \item cool to  $25\degree C$ at a rate $0.1\degree C/min$ and hold.
\end{enumerate}
\begin{figure}

\end{figure}

Once the device reaches $50\degree C$, a small drop of MBBA liquid crystal is carefully placed on the cell edge 
using a lab spatula. As the MBBA warms and becomes transparent, it is gently pushed toward the cell center, 
where capillary action draws it throughout the cell gap. The lid is closed, and the cell is left undisturbed 
for at least 24 hours during the slow cooling protocol, allowing the liquid crystal to reach the nematic phase 
and establish uniform homeotropic alignment with minimal defects. Upon completion, the cell is inspected under 
crossed polarizers to verify uniform alignment and is then ready for experimental use or stored in a sealed 
container at room temperature.


\begin{figure}[h]
    \centering
    \includegraphics[width=0.4\linewidth]{figs/imported/conoscopic_2_50X.png}
    \caption{Conoscopic image confirming the homeotropic alignment of the liquid crystal cell.}
    \label{fig:conoscopicr}
\end{figure}

\subsection{Optical microscopy system}
Electrodes from the function generator, which delivers an RMS AC sine wave voltage ranging from 10V to 60V at 
10kHz via an $A800DI$ high voltage linear amplifier, were connected to the cell. At a controlled temperature of 
$28\degree C$, dynamic data of defect dynamics were recorded under 
$10\times$ magnification for RMS voltages of 10V, 20V, 40V, and 60V. For defect formation studies, an electric-
field ramp was implemented using a 10 kHz sawtooth waveform, with the peak-to-peak voltage $V_{\mathrm{pp}}$ 
systematically varied across 200, 300, 400, 500, 600, 700, 800, and 900 mV to enable controlled traversal of 
the instability threshold.

\newpage
\section{Computational Framework}\label{sec:Comp} %------------------------------------------------------------

\indent Captured frame images are subsequently processed through a series of image analysis operations using 
the Python Sci-kit Image \cite{VanDerWalt2014Scikit-image:Python} package. The detailed analysis framework 
\footnote{Framework scripts, notebooks and analysis results can be found here: \href{https://github.com/mandal-anik10/SLP-CosmoWithLCs}{Repository}\\
\textbf{Clone:} \$ git clone https://github.com/mandal-anik10/SLP-CosmoWithLCs.git} is described below.

\subsection{Pre-processing and noise reduction}
In the first step of image processing, we have implemented a $3\times3$ median filter to reduce noise and 
improve the overall quality of the image. The median filter is a non-linear filtering technique that 
replaces each pixel value with the median of the intensity values within its 3×3 neighborhood. Specifically, 
for every pixel, a $3\times3$ window is centered on it, and the nine pixel values within this window are 
sorted in ascending order. The median value from this sorted list is then assigned to the central pixel, 
effectively suppressing impulsive noise such as salt-and-pepper noise while preserving important image 
details. Unlike linear filters that tend to blur edges by averaging pixel values, the median filter 
maintains sharp boundaries and fine structures, making it particularly effective for applications where edge 
preservation is crucial.
\begin{figure}[h]
    \centering
    \includegraphics[width=1.0\linewidth]{figs/test-framework-1_50s_28_60V_5-S012.png}
    \caption{Images after $3\times3$ median noise filter and Meijering ridge detection. The sample image is a single frame from 60V at $10\times$ magnification.} 
    \label{fig:test_fw_012}
\end{figure}

\subsection{Ridge detection operation}
For ridge detection, we implement the Meijering ridge detection filter \cite{Meijering2004DesignImages}, a 
modified Hessian-based approach specifically designed to detect elongated, tubular structures in noisy 
images. In this implementation, the required second-order image derivatives $f_{ij}(\mathbf{x})$ are 
computed by convolving the image $f$ with the second-order derivatives of a normalized Gaussian kernel $G$, 
expressed as $f_{ij}(\mathbf{x}) = (f \ast G_{ij})(\mathbf{x})$, where $G_{ij}(\mathbf{x}) = \left( 
\frac{\partial^{2}}{\partial_{i} \partial_{j}} G \right)(\mathbf{x})$ and $\mathbf{x} = (x, y)$ denotes the 
pixel position. The eigenvectors and eigenvalues are determined not from a standard Hessian, but from a 
modified matrix $\mathbf{H}'_{f}(\mathbf{x})$:

$$\mathbf{H}'_{f}(\mathbf{x}) = 
\begin{bmatrix} 
f_{xx}(\mathbf{x}) + \alpha f_{yy}(\mathbf{x}) & (1 - \alpha)f_{xy}(\mathbf{x}) \\ (1 - \alpha)f_{xy}
(\mathbf{x}) & f_{yy}(\mathbf{x}) + \alpha f_{xx}(\mathbf{x}) 
\end{bmatrix}$$

where $\alpha$ is a weighting parameter (typically set to $1/3$ for 2D images 
(\href{https://github.com/scikit-image/scikit-image/blob/e8a42ba85aaf5fd9322ef9ca51bc21063b22fcae/skimage/filters/ridges.py#L76}{Source})) that ensures 
the filter is maximally flat in the longitudinal direction of the ridge. The normalized eigenvectors 
$\mathbf{v}'_{i}(\mathbf{x})$ of this modified matrix remain identical to the standard Hessian eigenvectors 
$\mathbf{v}_{i}(\mathbf{x})$, while the modified eigenvalues $\lambda'_{i}(\mathbf{x})$ are calculated as 
$\lambda'_{1}(\mathbf{x}) = \lambda_{1}(\mathbf{x}) + \alpha \lambda_{2}(\mathbf{x})$ and $\lambda'_{2}
(\mathbf{x}) = \lambda_{2}(\mathbf{x}) + \alpha \lambda_{1}(\mathbf{x})$. The filter assigns a 
``neuriteness'' or ``vesselness'' measure $\Phi(x)$ to each pixel according to $\Phi(x) = 
\lambda(x)/\lambda_{\text{min}}$ for $\lambda(x) \geq 0$ and $\Phi(x) = 0$ otherwise, where $\lambda$ 
represents the larger modified eigenvalue in magnitude and $\lambda_{\text{min}}$ is the minimum eigenvalue 
across all pixels. By utilizing the eigenvector corresponding to the smaller absolute eigenvalue to 
indicate the longitudinal direction, this method effectively suppresses responses to background intensity 
discontinuities while enhancing continuous, filamentous features. This multi-scale approach proves superior 
to basic Hessian or Laplacian-based detectors for identifying nematic liquid crystal defects, as it 
maintains connectivity across regions of varying contrast and width while significantly reducing false 
detections from noise artifacts.

\subsection{Calculating defect density}
A sequence of processing operations was applied to the ridge maps to robustly isolate defect lines from the 
background and to accurately quantify their total length and resulting line density.

\subsubsection{Hysteresis thresholding:}
To distinguish true ridge structures from spurious noise artifacts in the ridge-detected images, we apply
hysteresis thresholding with two carefully selected thresholds, 
$T_{\text{high}}$\footnote{\label{config_file}Choosen thereshold parameter values and footprint size are given in 
the configuration file in the \href{https://github.com/mandal-anik10/SLP-CosmoWithLCs}{repository}.} and 
$T_{\text{low}}$\footref{config_file}, following the methodology established in the Canny edge detection framework 
\cite{Canny1986ADetection}. In this dual-threshold approach, pixels with ridge metric values above $T_{\text{high}}$ 
are immediately classified as definite defect pixels (strong edges), while those below $T_{\text{low}}$ are rejected 
as background. Critically, pixels with values between $T_{\text{low}}$ and $T_{\text{high}}$ (weak edge pixels) are 
retained only if they remain connected to already-classified defect pixels through 8-connectivity neighborhoods.

\begin{figure}[h]
    \centering
    \includegraphics[width=1.0\linewidth]{figs/test-framework-1_50s_28_60V_5-S345.png}
    \caption{Images after hysteresis thresholding, binary closing, removing small objects, and skeletonizing. The sample image is a single frame from 60V at $10\times$ magnification.}
    \label{fig:test_fw_345}
\end{figure}
\subsubsection{Binary closing and removing small objects:}
To further refine the segmentation, binary morphological closing\footref{config_file}  (a dilation followed by 
an erosion) is applied to bridge small gaps and holes within detected defect regions, followed by connected 
component analysis to remove small objects below a chosen length threshold ($\textbf{64px}$); together, these 
steps ensure that only large, continuous defect structures are retained while isolated noise pixels and 
spurious fragments are effectively suppressed, resulting in cleaner and more accurate representations of the 
underlying defect network.

\subsubsection{Skeletonizing:}
The refined binary image containing identified defect regions is processed through skeletonization, a 
morphological thinning operation that reduces each detected defect lines to its medial axis; a connected, one-
pixel-wide representation of the original structure. The skeletonization algorithm iteratively removes 
boundary pixels from the binary image while preserving connectivity and topological properties, ensuring that 
the reduced skeleton maintains the same homotopy as the original object. This compact representation enables 
precise measurement of defect lengths and facilitates reliable analysis of defect network
topology, even when original regions are wide or noisy.

To quantify the defect network from processed images, the total defect length is estimated by extracting the 
skeletonized form of each defect line and analyzing it with the \textit{Skan} \cite{Skeleton0.13.0} library, 
which reconstructs the skeleton as a network graph and computes the sum of branch lengths using pixel 
connectivity and spacing. The cumulative length of all skeleton branches in the field of view directly yields 
the projected defect length $L$. To obtain the defect density $\rho$, this total length is divided by the 
sample volume $V$ (the area imaged multiplied by the cell thickness), resulting in $\rho = L / V$. 

\newpage

\section{Defect Dynamics}\label{sec:Result}
In the present study, the Fréedericksz transition was deliberately induced in a homeotropically aligned MBBA nematic 
liquid crystal cell in order to investigate defect formation and subsequent coarsening dynamics under an applied 
electric field. By using the electric field as a controlled external parameter, the transition provides a well-
defined route to drive the system out of equilibrium and probe the emergence of topological defects. 

\begin{figure}[h]
\centering
\includegraphics[width=0.8\linewidth]{figs/imported/Fréedericksz_Transition.png}
\caption{Fréedericksz transition in a homeotropically aligned nematic liquid crystal (\href{https://doi.org/10.1039/B308098F}{Source}).}
\label{fig:Fréedericksz_Transition}
\end{figure}

\subsection{The Fréedericksz transition}
The Fréedericksz transition is a fundamental field-induced reorientation phenomenon in nematic liquid crystals, in 
which an initially uniform director configuration becomes unstable when an external electric or magnetic field 
exceeds a critical threshold. In a homeotropically aligned cell, the director is anchored perpendicular to the 
confining substrates in the absence of an applied field, resulting in a homogeneous alignment normal to the cell 
planes. When a destabilizing external electric field is applied, the torque exerted by the field on the anisotropic 
nematic medium competes with the elastic restoring forces associated with splay, twist, and bend deformations. Once 
the field strength exceeds the Fréedericksz threshold, the elastic energy can no longer sustain the uniform 
homeotropic state, and the director undergoes a continuous, spatially varying reorientation, developing in-plane 
components across the cell thickness (Figure~\ref{fig:Fréedericksz_Transition}).

The nature of this field-induced reorientation depends critically on the dielectric anisotropy of the nematic, 
defined as $\Delta\varepsilon = \varepsilon_{\parallel} - \varepsilon_{\perp}$. For materials with positive 
dielectric anisotropy ($\Delta\varepsilon > 0$), the director tends to align parallel to the applied electric field 
to minimize dielectric energy. In contrast, for materials with negative dielectric anisotropy ($\Delta\varepsilon < 
0$), the director prefers to orient perpendicular to the applied field.

MBBA is a nematic liquid crystal with negative dielectric anisotropy; therefore, an applied electric field 
destabilizes the initial homeotropic alignment by favoring director orientations perpendicular to the field. This 
setup serves as an ideal model for spontaneous symmetry breaking. Initially, the molecules stand vertically, 
possessing perfect rotational symmetry. When the applied voltage exceeds the critical threshold ($V > V_c$), this 
alignment becomes unstable. In a characteristic second-order transition, the magnitude of the director tilt 
increases continuously above this threshold. The instability is analogous to a ball balanced on the peak of a 
``Mexican Hat'' potential, while the molecules are energetically compelled to tilt, every azimuthal direction is 
equally valid. Consequently, random thermal fluctuations cause separate regions to tilt in different 
directions, and the boundaries where these mismatched domains meet are where topological defects form.

\begin{figure}[h]
    \centering
    \includegraphics[width=\linewidth]{figs//imported/pi_wall.png}
    \caption{(Left) Observed under a cross polarizer, a pair of $\pm1/2$ wedge disclinations connected by a $\pi$-wall when an in-plane electric field is present. (Right) Sketch of the 2D director field; $\theta$ is the angle between the $\pi$-wall and the electric field. (Source-\cite{Blanc2005DynamicsBackflow})}
    \label{fig:pi-wall}
\end{figure}

For such an electric-field-driven transition in a homeotropic nematic cell, the critical voltage $V_c$ is given, 
within the one-constant approximation, by

\begin{equation}
V_c = \pi \sqrt{\frac{K}{\varepsilon_0 |\Delta \varepsilon|}},
\end{equation}

where $K$ is the Frank elastic constant, $\varepsilon_0$ is the vacuum permittivity, and $\Delta \varepsilon$ is the 
dielectric anisotropy of the liquid crystal. 

This field-induced reorientation produces pronounced changes in optical birefringence and texture. The controllable 
instability associated with the Fréedericksz transition therefore provides an effective experimental parameter for 
probing defect formation and non-equilibrium dynamics in nematic liquid crystals.

\subsection{Late-time coarsening dynamics}

\begin{figure}[h]
\centering
\subfloat[Late-time coarsening dynamics for 10V]{\includegraphics[width=0.45\textwidth]{figs/10V-rho_dyn.png}}
\hspace{0.05\textwidth}
\subfloat[Late-time coarsening dynamics for 20V]{\includegraphics[width=0.45\textwidth]{figs/20V-rho_dyn.png}}
\vspace{0.5cm}
\subfloat[Late-time coarsening dynamics for 40V]{\includegraphics[width=0.45\textwidth]{figs/40V-rho_dyn.png}}
\hspace{0.05\textwidth}
\subfloat[Late-time coarsening dynamics for 60V]{\includegraphics[width=0.45\textwidth]{figs/60V-rho_dyn.png}}
\caption{\textbf{Old cell with permanent defects}: Log-log plot showing the late-time coarsening dynamics of defect density ($\rho$) versus time ($t$) for the 10V, 20V, 40V, 60V applied field case, observed at $10\times$ magnification. The solid orange line represents the estimated defect length. The green line depicts the best power-law fit within the scaling regime.}
\label{fig:old_dyn_string_density}
\end{figure}

\subsubsection*{Observation and interpretation : }

In the initial phase of this study, the prepared nematic liquid crystal cell (thickness $31.09 \pm 0.44\,\mu$m) 
exhibited a significant population of permanent defects resulting from the sample preparation process. These pre-
existing singularities served as effective nucleation sites for field-induced topological structures upon the 
application of the electric field. Consequently, the resulting defect network was characterized by the emergence of 
point defects with strength $\pm 1/2$, interconnected by $\pi$-walls (inversion walls), as illustrated in 
Figure~\ref{fig:pi-wall} and Figure~\ref{fig:old_60V_frames}.

\begin{figure}
    \centering
    \includegraphics[width=\linewidth]{figs//imported/domain_dyn.png}
    \caption{Shape evolution of a single domain at different moments and the corresponding polarizing optical microscope micrographs. (a) At the early stage of the annihilation, where the loop changes its shape drastically. (b) At the late stage of the annihilation, where the loop shrinks slowly with its shape unchanged. Black arrows indicate the velocities of different segments of the loop. The color of the loop indicates the local curvature. The color bar gives a linear scale of curvature $\kappa$. $\kappa_{min} = -1.24 \mu m^{−1}$ and $\kappa_{max} = 0.49 \mu m^{-1}$. (Source-\cite{Shen2020AnnihilationCrystals})}
    \label{fig:domain_dyn}
\end{figure}

The subsequent evolution of this network was governed primarily by the elastic interactions and mutual annihilation 
of these line defects. Analysis of the late-time coarsening dynamics yielded scaling exponents of $-1.20$ at $10$V, 
$-0.80$ at $20$V, $-0.98$ at $40$V, and $-1.02$ at $60$V (see Figure~\ref{fig:old_dyn_string_density}). These values 
are in good agreement with the theoretical prediction of inverse-time scaling ($t^{-1}$) characteristic of string-
dominated dynamics. The notable deviations observed at lower voltages ($10$V and $20$V) can be attributed to reduced 
optical contrast and increased ``fuzziness'' of the defect lines, which compromised the robustness of the line-
detection algorithm in this regime.

Regarding the uncertainty analysis for this dataset, it is important to note two limitations. First, random 
statistical errors could not be reliably estimated, as the data represents a single experimental realization for 
this sample. Second, uncertainties associated with the time variable were not accounted for in the extraction of the 
scaling exponents.\\

To isolate the intrinsic defect formation and evolution dynamics, the experiment was repeated using a newly 
prepared liquid crystal cell with no observable permanent defects. In the absence of predefined nucleation 
sites, the system predominantly generated closed-loop, domain-wall-like defects in order to conserve 
topological charge. Such loop-dominated defect networks are expected to exhibit slower coarsening dynamics, 
with a theoretical scaling exponent close to $-0.5$ \cite{Shen2020AnnihilationCrystals}. However, the measured 
exponents were substantially smaller ($20$V: $-0.21$, $40$V: $-0.22$, $60$V: $-0.33$), indicating that the early and 
intermediate-time dynamics are governed by fluctuation-dominated loop evolution rather than simple curvature-driven 
collapse.

\begin{figure}[h]
    \centering
    \subfloat[Late-time coarsening dynamics for 10V]{\includegraphics[width=0.45\textwidth]{figs/28_new_10V_scalling.png}}
    \hspace{0.05\textwidth}
    \subfloat[Late-time coarsening dynamics for 20V]{\includegraphics[width=0.45\textwidth]{figs/28_new_20V_scalling.png}}
    \vspace{0.5cm} 
    \subfloat[Late-time coarsening dynamics for 40V]{\includegraphics[width=0.45\textwidth]{figs/28_new_40V_scalling.png}}
    \hspace{0.05\textwidth} 
    \subfloat[Late-time coarsening dynamics for 60V]{\includegraphics[width=0.45\textwidth]{figs/28_new_60V_scalling.png}}
    \caption{ \textbf{New cell without permanent defects}: Log-log plot showing the late-time coarsening dynamics of defect density ($\rho$) versus time ($t$) for the 10V, 20V, 40V, 60V applied field case, observed at $10\times$ magnification. The solid blue line represents the average of the estimated defect length, with the surrounding shaded region indicating the $1\sigma$ limits of random error. The orange line depicts the best power-law fit within the scaling regime. The shaded red regions mark data excluded from the analysis: early-time data (left) were discarded to remove the initial 10–25 frames, while the late-time tail (right) was excluded based on relative error selection criteria.}
    \label{fig:dyn_string_density}
\end{figure}

At higher applied voltages, the strong dielectric torque confines the director more rigidly along the field 
direction, stabilizing the closed-loop topology and suppressing interactions with the cell boundaries. 
Consequently, these loop structures persist for extended durations (up to $\sim 15$–$20$s), as illustrated in 
figures-\ref{fig:20V_frames}, \ref{fig:40V_frames}, and \ref{fig:60V_frames}. At much later times ($\gtrsim 
20$s), the loops occasionally open, most likely due to residual interactions with the substrates or weak 
quenched disorder in the bulk. This opening process converts closed loops into point defects connected by 
$\pi$-wall, triggering a crossover to faster, string-dominated coarsening dynamics. However, by this stage, the 
defect density is already very low, rendering statistical estimates of the scaling exponent unreliable.

In contrast, at $10$V the weaker electric field provides less dielectric stabilization, allowing enhanced 
director fluctuations and earlier interactions with the cell boundaries. As a result, closed loops open at 
comparatively early times (figure-\ref{fig:10V_frames}), leading to a premature crossover to line-defect 
dynamics and a corresponding increase in the observed scaling exponent to $-0.74$. Overall, these results show 
that the applied voltage controls how defect dynamics evolve, causing a transition from closed-loop defects to 
open line defects. This crossover is governed by the strength of the electric field, the presence of quenched 
disorder, and interactions with the cell boundaries, all of which determine the dominant coarsening mechanism 
and the observed scaling behavior.

\subsection{Defect formation dynamics}

\subsubsection*{Observation and interpretation : }

\begin{figure}
    \centering
    \includegraphics[width=\linewidth]{figs/formation_dyn.png}
    \caption{Analysis result from formation dynamics study. Defect densities have been plotted against the applied electric field ramp rate at 4 different initial time stamps.}
    \label{fig:formation_dyn}
\end{figure}

For defect formation studies, an electric-field ramp was implemented using a $10\,\text{kHz}$ sawtooth 
waveform, with the peak-to-peak voltage ($V_{pp}$) systematically varied across $200$, $300$, $400$, $500$, 
$600$, $700$, $800$, and $900\,\text{mV}$ to enable controlled traversal of the instability threshold. However, 
at this high frequency, the oscillation period of the electric field is significantly shorter than the 
characteristic relaxation time of the nematic director. Consequently, the system cannot react effectively to 
the instantaneous peak voltage, as the molecules are unable to reorient within the rapid oscillation cycles. 
This timescale mismatch causes the director field to fail to adiabatically follow the ramp, leading to an 
effective ``freeze-out" of the dynamics. Instead of evolving through a critical phase transition, the system 
gets locked into a non-equilibrium state at a random finite length scale. As a result, the final defect 
structure is ``immature," governed principally by stochastic local factors rather than intrinsic physical laws. 
This mechanism explains the significant scatter in the defect density data observed in figure-
\ref{fig:formation_dyn} and the absence of the power-law scaling expected from the Kibble-Zurek mechanism.

To overcome this freeze-out and allow the liquid crystal molecules sufficient time to respond to the applied 
field, the driving frequency was reduced. However, at lower frequencies (1–10 Hz), mobile ions within the 
liquid crystal are able to drift and accumulate, forming regions of excess charge. These charges interact with 
the electric field and drive fluid motion inside the cell. The resulting flow strongly distorts the molecular 
alignment, producing irregular and highly fluctuating optical textures that disrupt the ordered defect 
structures (as shown in the figure-\ref{fig:electroconvection}).

\begin{figure}
    \centering
    \includegraphics[width=0.5\linewidth]{figs/electro_convection_2025-12-01T20-15-08.641.png}
    \caption{Electroconvection observed in the prepared MBBA Liquid crystal cell at applied electric field frequency of order $\sim Hz$ and applied voltage of order $\sim10V$.}
    \label{fig:electroconvection}
\end{figure}

\subsection{Error analysis}

\subsubsection{Error in time}

Temporal uncertainty arises from a significant $\textbf{30\%}$ frame drop rate observed during the 
$100\,\text{fps}$ recording process. This data loss is attributed to hardware bandwidth limitations, where the 
data write speed was insufficient to sustain the continuous storage of the high-frequency video stream. To 
reconstruct the true temporal evolution, a Monte Carlo resampling method was employed. The observed frame 
sequence was sequentially sampled $1000$ times in the corrected time range (scaled by $100/70$) to account for 
the stochastic nature of the drops. For each frame $i$, the median ($\mu_{t_i}$) of the simulated arrival times 
was assigned as the corrected timestamp, while the standard deviation ($\sigma_{t_i}$) represents the temporal 
error:

\[
    t_i = \mu_{t_i} \pm \sigma_{t_i}
\]

This statistical reconstruction mitigates the non-linear distortion introduced by the write-speed bottleneck.

Although the temporal uncertainty $\sigma_{t_i}$ was rigorously calculated for every data point, horizontal 
error bars are omitted from the final plots to maintain graphical clarity. However, the analysis explicitly 
incorporates the corrected timestamps ($\mu_{t_i}$) derived from the resampling procedure for the time axis. 
This ensures that the reported dynamics are grounded in the statistically reconstructed timeline, effectively 
acknowledging the latency-induced distortions.

\subsubsection{Error in defect density}

\noindent{\fontsize{12pt}{12pt}\selectfont \textbf{Systematic errors:}} The accuracy of the string defect 
density measurement is fundamentally limited by the precision of the geometric parameters used in the 
capacitance method. Specifically, a unit length measurement error of $2\,\text{mm}$ introduces an uncertainty 
of $4\,\text{mm}^2$ in the unit area determination. Given the total measured sample area($A$) of 
$80\,\text{mm}^2$, this corresponds to a relative fractional error of $5\%$. This geometric uncertainty 
propagates linearly through the analysis; the fractional error in the area measurement translates directly to 
the uncertainty in the calculated cell thickness ($t$) and subsequently to the string defect density ($\rho$). 
Consequently, the system exhibits a uniform error propagation described by the relationship:

\[
    \frac{dA}{A} = \frac{dt}{t} = \frac{d\rho}{\rho} \approx 0.05
\]

Further, the pixel-to-length calibration was established by equating a physical length of $80\,\text{mm}$ to 
$1174\,\text{pixels}$. The precision of this calibration is limited by the finite thickness of the scale 
notation bars, which exhibit a standard deviation width of approximately $1.8\,\text{pixels}$. Accounting for 
the measurement uncertainty at both endpoints of the scale, the total pixel error accumulates to $\approx 
3.6\,\text{pixels}$. Consequently, the relative error in the calibration factor($C$) is estimated as:

\[
    \frac{\delta C}{C} = \frac{\delta L_{\text{px}}}{L_{\text{px}}} = \frac{3.6}{1174} \approx 0.3\%
\]

This calibration uncertainty propagates to the string density calculation. Since the calculated cell volume (V) 
scales with the square of the calibration factor ($V \propto C^2$) while the measured string length scales 
linearly ($L \propto C$), the defect density depends inversely on the calibration factor ($\rho \propto 
C^{-1}$). Consequently, the magnitude of the relative error in density is equivalent to that of the calibration 
factor:

\[
    \frac{d\rho}{\rho} \approx \frac{dC}{C} \approx 0.3\%
\]

This results in a total systematic uncertainty of $5.3\%$ in the estimated string density.

\noindent{\fontsize{12pt}{12pt}\selectfont \textbf{Random error:}} To quantify the random error inherent in the 
experimental setup, five (three in the case of defect formation study) independent measurements were conducted 
on the same sample under identical conditions. The observed variations in the measured values are attributed to 
stochastic fluctuations in the detection mechanism and manual selection processes. These datasets were analyzed 
to compute the arithmetic mean ($\bar{x}$) and the standard deviation ($\sigma$), which serves as the primary 
metric for the random uncertainty. The final experimental result is reported as the mean with the standard 
deviation representing the measurement precision:

\[
    x = \bar{x} \pm \sigma = \frac{1}{N}\sum_{i=1}^{N} x_i \pm \sqrt{\frac{\sum_{i=1}^{N}(x_i - \bar{x})^2}{N-1}}
\]

where $N=5$ represents the number of trials. This statistical variation captures the reproducibility of the 
system, distinct from the systematic calibration errors discussed previously.

\begin{figure}[h]
    \centering
    \includegraphics[width=\linewidth]{figs/datapoint_selection_criteria.png}
    \caption{This example figure(60V data) shows the region selected based on the criteria given below. The shaded red regions mark data excluded from the analysis: early-time data (left) were discarded to remove the initial 10 frames, while the late-time tail (right) was excluded based on relative error selection criteria.}
    \label{fig:sel_criteria}
\end{figure}

Prior to analysis, the three independent measurement series were time-shifted to align with a common temporal 
reference point ($t=0$). To ensure the robustness of the subsequent model fitting, the dataset was filtered to 
exclude regimes of high variability. Specifically, the analysis interval was restricted to the range where the 
20-point median value of relative standard deviation remained below $\textbf{10\%}$ of the mean value:

\[
    Median \left(\frac{\sigma(t_{-10})}{\bar{x}(t_{-10})}, \frac{\sigma(t_{-9})}{\bar{x}(t_{-9})},...,\frac{\sigma(t_{0}=t)}{\bar{x}(t_{0}=t)},..., \frac{\sigma(t_{9})}{\bar{x}(t_{9})}\right)(t) < \textbf{0.1}
\]

This criterion is introduced to suppress spurious fluctuations and to retain the underlying median trend of the 
data. Data points that violate this stability condition are discarded to ensure that transient deviations do 
not bias the extracted scaling parameters.

\newpage

\section{Conclusion}\label{sec:Conc}

This project developed a reliable experimental and computational approach to study defect dynamics in nematic liquid 
crystals as a laboratory model for cosmological phase transitions. Although a direct experimental test of the 
Kibble–Zurek mechanism (KZM) scaling laws was not possible with the present setup, the work provided important 
insight into late-time defect evolution and the influence of topology and electric fields on nonequilibrium behavior.

In the late-time regime, a clear change in scaling behavior was observed, determined by the structure of the defect 
network. When permanent nucleation sites were present, the system evolved mainly through the annihilation of open 
line defects, and the defect density decayed as $\rho(t)\propto t^{-1}$, in agreement with standard one-scale 
coarsening theory. In contrast, in defect-free cells the system initially formed closed defect loops. Under strong 
electric fields, these loops were long-lived and decayed much more slowly than expected from simple tension-driven 
models. Additionally, variations in the $\pi$-wall width were observed during these dynamics, consistent with the 
expected influence of the electric field on the coherence length.

Efforts to study KZM scaling during defect formation showed that the experiment is limited by two extreme regimes. 
At high driving frequencies ($10\,\mathrm{kHz}$), the electric field changes too quickly for the liquid crystal 
molecules to respond, causing the system to freeze before critical dynamics can develop. At lower frequencies, where 
molecular relaxation is possible, ion motion generates strong fluid flow that disrupts the ordered defect patterns. 
Because of this, no stable intermediate frequency range exists in the present sample configuration where the 
predicted KZM defect-scaling laws can be clearly observed.

Overall, these results demonstrate the robustness of late-time coarsening behavior in nematic liquid crystals while 
revealing practical limits imposed by material response times and electrohydrodynamic effects. Future studies could 
overcome these limitations by employing magnetic field quenches or rapid temperature quenches with suitable nematic 
materials to access universal defect-formation scaling. Furthermore, the observation of variable $\pi$-wall widths 
opens the door to quantitatively testing the inverse scaling of width with applied field strength, as predicted in 
literature\cite{Blanc2005DynamicsBackflow}. Finally, the complex annihilation dynamics of umbilic defects,
specifically their predicted S-shaped trajectories, could be resolved with much greater precision by applying 
sophisticated machine learning approaches to defect tracking\cite{Dierking2025MachineNanoparticles}.

\appendix
\section{Acknowledgements}

I would like to express my sincere gratitude to Prof. Pramoda for providing me with the opportunity to 
conduct this research in his laboratory. I am also deeply thankful to Ankit for his invaluable 
mentorship and unwavering support throughout the study.

I extend my gratitude to Prof. Somak and my SRC members for encouraging me to pursue this semester-long 
project and for granting me the academic autonomy to explore these ideas.

I also wish to thank Shri Gowri, Kurshid da, and all other members of our small physics family for the 
fruitful discussions and for helping me gain conceptual clarity. Their willingness to assist whenever I 
needed help was essential to the completion of this project.

I gratefully acknowledge the Departments of Chemistry and Biology for providing access to their 
laboratory facilities.

Finally, I would like to acknowledge the developers of the open-source software ImageJ and other scientific analysis 
tools (e.g. scikit-image) that significantly streamlined this work. Specifically, I thank the creators of GitHub for 
managing project repositories, Overleaf for the seamless typesetting environment, and Mendeley for bibliography 
management. I also appreciate the utility of generative artificial intelligence tools, particularly for 
their artificial intelligence-assisted search prompts, which accelerated the literature review and 
documentation process.


\bibliographystyle{naturemag}
\bibliography{references}


\section{Late-time coarsening}
\begin{figure}
    \centering
    \includegraphics[width=\linewidth]{figs/old_60V_dyn.png}
    \caption{Topological defects at different time stamps for 60V $10\times$ (Old liquid crystal cell of thickness $31.09\pm0.44\mu m$ with permanent defect points). Red curves are the defect lines detected by the implemented framework.}
    \label{fig:old_60V_frames}
\end{figure}

\begin{figure}
    \centering
    \includegraphics[width=\linewidth]{figs/28_new_10V_dyn.png}
    \caption{Topological defects at different time stamps for 10V $10\times$. Red curves are the defect lines detected by the implemented framework.}
    \label{fig:10V_frames}
\end{figure}

\begin{figure}
    \centering
    \includegraphics[width=\linewidth]{figs/28_new_20V_dyn.png}
    \caption{Topological defects at different time stamps for 20V $10\times$. Red curves are the defect lines detected by the implemented framework.}
    \label{fig:20V_frames}
\end{figure}

\begin{figure}
    \centering
    \includegraphics[width=\linewidth]{figs/28_new_40V_dyn.png}
    \caption{Topological defects at different time stamps for 40V $10\times$. Red curves are the defect lines detected by the implemented framework.}
    \label{fig:40V_frames}
\end{figure}

\begin{figure}
    \centering
    \includegraphics[width=\linewidth]{figs/28_new_60V_dyn.png}
    \caption{Topological defects at different time stamps for 60V $10\times$. Red curves are the defect lines detected by the implemented framework.}
    \label{fig:60V_frames}
\end{figure}
\end{document}