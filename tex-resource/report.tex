\documentclass[12pt]{article}
\usepackage[utf8]{inputenc}

\usepackage[T1]{fontenc}
\usepackage{amsmath,amsfonts,amssymb}
\usepackage{geometry, graphicx, float, hyperref, fancyhdr, titlesec, setspace, abstract, indentfirst, booktabs, multirow, array, caption, subcaption, fontspec, enumitem}
\setlength{\headheight}{14.5pt}
\usepackage[dvipsnames]{xcolor}


% Page setup
\pagestyle{fancy}
\fancyhf{}
\addtolength{\headwidth}{\marginparwidth}
\fancyhead[R]{\thepage}
\fancyhead[L]{\leftmark}
\renewcommand{\headrulewidth}{1pt}

\geometry{
    a4paper,
    left=2.4cm,
    right=2.4cm,
    top=3.0cm,
    bottom=2.5cm
}

% Hyperref setup
\hypersetup{
    colorlinks=true,
    linkcolor=RoyalBlue,
    filecolor=magenta,      
    urlcolor=RoyalBlue,
    pdftitle={Laboratory Studies of Cosmology-Inspired Defect Dynamics in Liquid Crystals},
    pdfauthor={Anik Mandal},
    bookmarksopen=true,
    bookmarksnumbered=true
}

% Custom commands
\newcommand{\email}[1]{\texttt{#1}}
\newcommand{\university}{Ashoka University}
\newcommand{\location}{Sonipat, India}


\begin{document}

% Title page
\begin{titlepage}
    \centering
    \vspace*{2cm}
    {\huge\bfseries Laboratory Studies of Cosmology Inspired Defect Dynamics in \\Liquid Crystals\par}
    \vspace{2cm}
    {\Large Semester Long Project Report\par}
    \vspace{1.5cm}
    {\large
    \textbf{Anik Mandal}$^{1,*}$ \\[0.5em]
    SLP Guide: Prof. Pramoda Kumar$^1$ \\
    Supervisor: Prof. Somak Raychaudhury$^1$ \\[1em]
    $^1$\university, \location \\
    $^{*}$Email: \href{mailto:anik.mandal_phd24@ashoka.edu.in}{anik.mandal\_phd24@ashoka.edu.in}
    }
    \vfill
    {\large \today\par}
\end{titlepage}

% Abstract
\begin{abstract}

\end{abstract}

% Table of Contents
\setcounter{tocdepth}{2}
\tableofcontents
\newpage

% Main content
\section{Introduction}


\subsection{Kibble-Zurek Mechanism}

\subsection{Topological Defects in Cosmology}

\subsection{Project Objectives}

\section{Experimental Setup}
\subsubsection*{Nematic Liquid Crystals}

\subsubsection*{Homeotropic Alignment}
% \subsection{apparatus}
\subsection{Preparing empty cell}
\subsubsection*{\textit{Patterning:}}
\subsubsection*{\textit{Etching:}}
\subsubsection*{}
\subsection{Capacitance-Based Cell Thickness Measurement}
\subsection{Preparing homeotropic cell}

\subsection{Optical Microscopy System}

\section{Computational Framework}
\subsection{String density estimation}
\subsubsection*{Step-1: Pre-processing and Noise Reduction}
In the first step of image processing, we have implemented a $3\times3$ median filter to reduce noise and improve the 
overall quality of the image. The median filter is a non-linear filtering technique that replaces each pixel value 
with the median of the intensity values within its 3×3 neighborhood. Specifically, for every pixel, a $3\times3$ 
window is centered on it, and the nine pixel values within this window are sorted in ascending order. The median 
value from this sorted list is then assigned to the central pixel, effectively suppressing impulsive noise such as 
salt-and-pepper noise while preserving important image details. Unlike linear filters that tend to blur edges by 
averaging pixel values, the median filter maintains sharp boundaries and fine structures, making it particularly 
effective for applications where edge preservation is crucial.

\subsubsection*{Step-2: Ridge detection operation}
\subsubsection*{Step-3: Hysteresis thresholding} 
\subsubsection*{Step-4: Binary closing and Removing small objects}
\subsubsection*{Step-5: Skeletonizing}
\subsubsection*{Step-6: Calculating string density}

\subsection{}



\section{Defect Dynamics}
\subsubsection*{The Freedericksz Transition}
\subsection{Electric Field Quench}
\subsection{Magnetic Field Quench}

\section{Conclusion}

% Bibliography
\bibliographystyle{naturemag}
\bibliography{references}

% Appendices (if needed)
\appendix
\section{Supplementary Materials}

% Add any supplementary materials, equations, or detailed procedures here

\end{document}