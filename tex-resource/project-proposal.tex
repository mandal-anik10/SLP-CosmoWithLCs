

\documentclass[12pt]{article}
\usepackage[utf8]{inputenc}

\usepackage{geometry, hyperref, amsmath, fontspec, enumitem}
\usepackage[dvipsnames]{xcolor}

\setmainfont{Georgia}
\geometry{
    a4paper,
    left=1.8cm,
    right=1.8cm,
    top=1.8cm,
    bottom=1.8cm
}
\hypersetup{colorlinks=true, linkcolor=RoyalBlue, urlcolor=RoyalBlue}

\begin{document}
\noindent
{\LARGE Project Proposal\hrulefill}\\[5pt]
{\Large \textbf{Laboratory Studies \cite{Chuang1991CosmologyCrystals,Chuang1991Late-timeCrystal,Kibble2007Phase-transitionUniverse} of Cosmology inspired Defect Dynamics in Liquid Crystals}}\\[5pt]
Anik Mandal$^{1,*}$ \\
SLP guide : Prof. Pramoda Kumar$^{1}$\\
Supervisor : Prof. Somak Raychaudhury$^{1}$\\[0.5em]
\small
$^{1}$Ashoka University, Sonipat, India \\[0.5em]
$^{*}$Email: \href{mailto:anik.mandal_phd24@ashoka.edu.in}{anik.mandal\_phd24@ashoka.edu.in}
\\
\rule{\linewidth}{0.4pt}

\begin{center}
    \textbf{Abstract}\\[5pt]
    \begin{minipage}{15cm}
    Nematic liquid crystals(NLCs) offer a unique laboratory platform for probing the dynamics of 
    topological defects analogous to cosmic strings, monopoles, and textures formed during symmetry-
    breaking phase transitions in the early universe. We will induce rapid isotropic–nematic transitions via controlled electric-field and temperature quenches and capture defect dynamics with high-speed polarized-optical videography. String production will be analyzed to validate both the Kibble mechanism of defect creation and its quench-rate prediction (density $\propto \tau_Q^{-1/2}$), while subsequent network coarsening will be quantified against the one-scale annihilation law (density $\propto t^{-1}$). Intercommutation events will be documented to characterize loop formation. In addition, we will explore texture decay-monopole-antimonopole pinch-off-through targeted experiments and one-constant relaxation simulations. This interdisciplinary investigation offers critical laboratory tests of cosmological defect theories and their implications for large-scale structure formation.
    \end{minipage}
\end{center}

\noindent{\fontsize{15pt}{12pt}\selectfont \textbf{Introduction}}\\

A symmetry-breaking phase transition is a fundamental physical phenomenon where a system undergoes
a transition from a more symmetric, higher-energy state to a less symmetric, lower-energy state. 
During this transition, the system’s symmetry present at higher temperatures or energies is 
spontaneously lost as it reaches a critical point, resulting in an ordered phase characterized by 
new properties. As the system transitions, different regions may choose different ways to break the 
symmetry independently, which can cause mismatches where these regions meet. These mismatches create 
defects, which are places in the material where the order is disrupted or the alignment is not well 
defined. The dynamical study of defects formed during symmetry-breaking phase transitions is 
increasingly important in cosmology, particle physics, and condensed matter physics.\\

The \textbf{Kibble-Zurek mechanism (KZM)} provides a unifying framework to describe defect formation 
during rapid symmetry-breaking transitions. In cosmology, as the early universe cooled from the Big 
Bang, the initially unified forces separated through successive symmetry-breaking phase transitions. 
Kibble proposed \cite{Kibble1976TopologyStrings} that this process occurred unevenly across space, giving rise to 
mismatches at causal boundaries that manifested as stable topological defects such as domain walls, 
cosmic strings, and monopoles. These defects are not only relics of the early universe but also 
form the theoretical basis for several models of cosmic structure formation \cite{Turok1989GlobalStructure}.\\

Zurek extended this cosmological idea to condensed matter systems, suggesting that analogous defects 
could form in laboratory settings \cite{Zurek1985CosmologicalHelium}. For example, a rapid phase transition in 
superfluid helium should generate vortex lines as the condensed-matter equivalents of cosmic 
defects. While practical challenges hindered experimental verification in helium, other systems, 
such as liquid crystals \cite{Gennes1993TheCrystals}, provide a more accessible platform. Their phase 
transitions occur at ordinary temperatures, and the resulting defect dynamics can be directly 
observed with optical microscopy, making them valuable analogs for probing the Kibble-Zurek 
mechanism.\\
\newpage
\noindent{\fontsize{15pt}{12pt}\selectfont \textbf{Project Objectives}}\\

This semester-long project aims to establish nematic liquid crystals (NLCs) as an experimental 
laboratory system for studying the dynamical evolution of topological defects (strings, monopoles, 
and textures) that are directly analogous to those predicted in cosmological theories. The primary 
objective is to  experimentally validate key theoretical predictions including the Kibble-Zurek 
mechanism of defect production during symmetry-breaking phase transitions, the “one-scale” scaling 
model for cosmic string evolution characterized by defect density decay following the annihilation 
law $\rho \propto t^{-1}$ with time $t$, and string intercommutation processes that occur when 
cosmic strings cross and reconnect. Additionally, we will test the KZM prediction \cite{Fowler2017KibbleZurekCrystal} 
$\rho \propto \tau_Q^{-1/2}$, by varying quench rates $\tau_Q$. We also plan to include an 
investigation of texture decay—generating $\pm1$ rings in suspended films and using simulations to 
track monopole–antimonopole formation, subject to project scheduling.\\

\noindent{\fontsize{15pt}{12pt}\selectfont \textbf{Methodology}}\\

The proposed study will build directly upon the experimental framework of \textit{Chuang et 
al.}\cite{Chuang1991CosmologyCrystals, Chuang1991Late-timeCrystal} by employing 4-cyano-4′-n-pentylbiphenyl (5CB) nematic liquid 
crystal as an accessible analog for cosmological defects and precisely replicating their temperature
quench apparatus to induce rapid isotropic-to-nematic transitions. Samples will be confined within a 
custom cell maintained at $33^{\circ}C$, will generate dense networks of strings and 
monopoles via the Kibble mechanism. A high-speed optical microscope will record the resulting defect 
tangle over intervals from 1 to 32 s, allowing automated image analysis to extract string density 
$\rho(t)$. \\

To test the Kibble–Zurek prediction of $\rho \propto \tau_Q^{-1/2}$ scaling law, we will perform 
controlled electric-field ``quenches” in NLC cells by varying the field-ramp time $\tau_Q$. To 
examine texture decay, $\pm1$ defect loops will be hand-drawn in freely suspended films using a 
micrometer-driven glass probe and their monopole-antimonopole pinch-off tracked under optical 
microscopy; simple relaxation simulations on a $40^3$ lattice in the one-constant approximation will 
replicate this process.\\

\noindent{\fontsize{15pt}{12pt}\selectfont \textbf{Expected Outcomes}}\\
\begin{itemize}[noitemsep, topsep=0pt, parsep=0pt, partopsep=0pt, itemsep=0pt]
    \item We are expecting to find direct visualization of monopoles and strings in nematic liquid 
    crystals, confirming their predicted structures and behaviors under unequal elastic constants.\\
    \item We are expecting to validate the one-scale scaling model for string network evolution, 
    with string density following a $\rho \propto t^{-1}$ power law.\\
    \item We are expecting to document string intercommutation events, supporting the reconnection 
    processes important for loop formation and network evolution.\\
    \item We are expecting to validate the Kibble-Zurek scaling law $\rho \propto \tau_Q^{-1/2}$, 
    demonstrating how defect density depends on the quench rate during phase transitions.\\
    \item We are expecting to observe texture decay through the formation and annihilation of 
    monopole–antimonopole pairs, providing the laboratory evidence of texture unwinding consistent 
    with simulations.\\
    \item We are expecting to measure texture decay rates and lifetimes that explain the rarity of 
    stable textures.\\
\end{itemize} 

These outcomes will demonstrate that nematic liquid crystals serve as effective experimental analogs 
for cosmological defect phenomena, offering critical tests of the Kibble mechanism, scaling laws, 
and topological defect decay relevant to large-scale structure formation.


\bibliographystyle{naturemag}
\bibliography{references}


\end{document}
