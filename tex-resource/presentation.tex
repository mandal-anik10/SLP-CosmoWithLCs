%============================================================================================================
%
%============================================================================================================

\documentclass[10pt]{beamer}
\usepackage{tikz}
\usetikzlibrary{calc,trees,positioning,arrows,chains,shapes.geometric,  decorations.pathreplacing,decorations.pathmorphing,shapes,  matrix,shapes.symbols}

\usetheme[sectionpage=none, numbering=fraction, progressbar=foot, block=fill, background=light ]{metropolis}
\useoutertheme{miniframes}
\setbeamerfont{footnote}{size=\scriptsize}
\addtobeamertemplate{footline}{\vspace{-1cm}}{}
\setbeamersize{text margin left=.5cm, text margin right=.5cm}

%=====| Adding custom thickness progress bar |===============================================================
\makeatletter
\setlength{\metropolis@progressinheadfoot@linewidth}{2pt} % Set thickness
\makeatother
\setbeamercolor{progress bar}{fg=red}

%=====| Adding section navigation bar at the top |===========================================================
\setbeamerfont{section in head/foot}{series=\bfseries}
\setbeamertemplate{headline}{%
  \begin{beamercolorbox}[wd=\paperwidth,ht=2ex,dp=1ex]{section in head/foot}
    \insertsectionnavigationhorizontal{\paperwidth}{}{}
  \end{beamercolorbox}
}

%=====| Adding custom reference in frames |==================================================================

\usepackage[backend=biber]{biblatex}
\addbibresource{references.bib}

\DeclareCiteCommand{\footcite}[\footnote]
  {\usebibmacro{prenote}}
  {\printnames{labelname}\space(\printfield{year})\space\printfield{doi}}
  {}
  {\usebibmacro{postnote}}



%=====| Adding custom Author–Date box in bottom-right corner |===============================================
\addtobeamertemplate{footline}{}{
  \begin{tikzpicture}[remember picture,overlay]
    \node[anchor=south east, xshift=-2pt, yshift=2pt] at (current page.south east) {
       \scalebox{.75}{\textcolor{gray}{\insertauthor; \insertdate}}
    };
  \end{tikzpicture}
}

% Title page informations ###################################################################################
\title{SLP Presentation}
\subtitle{Laboratory Studies of Cosmology-Inspired Defect Dynamics in Liquid Crystals}
\date{\today}
\author{Anik Mandal}
\institute{Ashoka University}
\titlegraphic{\hfill\includegraphics[height=1.5cm]{../tex-resource/ashoka-logo.png}}


\begin{document}

\maketitle
\begin{frame}{Project objectives}
\begin{itemize}
    \item akjd
    \item ajkdn
    \item skdjn
\end{itemize}
    
\end{frame}

\begin{frame}{Table of contents}
  \setbeamertemplate{section in toc}[sections numbered]
  \tableofcontents%[hideallsubsections]
\end{frame}


\section[Intro]{Introduction}

\begin{frame}{Symmetry-breaking in early Universe}
\scriptsize{abc}
    
\end{frame}

\begin{frame}[fragile]{Kibble-Zurek Mechanism}
hk\footcite{Chuang1991CosmologyCrystals}\footcite{Meijering2004DesignImages}\footcite{Palffy-Muhoray2007TheCrystals}
\end{frame}

\section{Experimental Setup}
\begin{frame}{Experimental Setup}

\begin{columns}

\begin{column}{0.25\textwidth}

\centering
\begin{tikzpicture}[
    node distance=0.25cm,
    every node/.style={draw, rectangle, rounded corners, minimum width=0.5cm, minimum height=0.5cm, text width=2.7cm, align=center},
    highlight/.style={draw=black, thick, fill=black!15, font=\small},
    faded/.style={draw=black!40, text=black!40, font=\scriptsize},
    arrow/.style={->, thick, draw=black!40}
]
\node[highlight] (1) {Patterning};
\node[highlight, below=of 1] (2) {Etching};
\node[highlight, below=of 2] (3) {Post etching treatment};
\node[highlight, below=of 3] (4) {Applying surfactant (DMOAP)};
\node[highlight, below=of 4] (5) {Cell assembly};
\node[highlight, below=of 5] (6) {Cell thickness measurement};
\node[highlight, below=of 6] (7) {LC filling and cooling};

\draw[arrow] (1) -- (2);
\draw[arrow] (2) -- (3);
\draw[arrow] (3) -- (4);
\draw[arrow] (4) -- (5);
\draw[arrow] (5) -- (6);
\draw[arrow] (6) -- (7);
\end{tikzpicture}

\end{column}

\begin{column}{0.75\textwidth}
Here you explain what happens at stage C.
You can add equations, figures, or intuition here.
\end{column}
\end{columns}
    
\end{frame}
% Patterning----------------------------------------------------------------------------------------
\begin{frame}{\small{Preparing Homeotropic Cell}}

\begin{columns}

\begin{column}{0.25\textwidth}

\centering
\begin{tikzpicture}[
    node distance=0.25cm,
    every node/.style={draw, rectangle, rounded corners, minimum width=0.5cm, minimum height=0.5cm, text width=2.7cm, align=center},
    highlight/.style={draw=black, thick, fill=black!15},
    faded/.style={draw=black!40, text=black!40, font=\scriptsize},
    arrow/.style={->, thick, draw=black!40}
]
\node[highlight] (1) {Patterning};
\node[faded, below=0.5cm of 1] (2) {Etching};
\node[faded, below=of 2] (3) {Post etching treatment};
\node[faded, below=of 3] (4) {Applying surfactant (DMOAP)};
\node[faded, below=of 4] (5) {Cell assembly};
\node[faded, below=of 5] (6) {Cell thickness measurement};
\node[faded, below=of 6] (7) {LC filling and cooling};

\draw[arrow] (1) -- (2);
\draw[arrow] (2) -- (3);
\draw[arrow] (3) -- (4);
\draw[arrow] (4) -- (5);
\draw[arrow] (5) -- (6);
\draw[arrow] (6) -- (7);
\end{tikzpicture}

\end{column}

\begin{column}{0.75\textwidth}
Here you explain what happens at stage C.
You can add equations, figures, or intuition here.
\end{column}
\end{columns}

\end{frame}

% Eatching------------------------------------------------------------------------------------------
\begin{frame}{\small{Preparing Homeotropic Cell}}

\begin{columns}

\begin{column}{0.25\textwidth}

\centering
\begin{tikzpicture}[
    node distance=0.25cm,
    every node/.style={draw, rectangle, rounded corners, minimum width=0.5cm, minimum height=0.5cm, text width=2.7cm, align=center},
    highlight/.style={draw=black, thick, fill=black!15},
    faded/.style={draw=black!40, text=black!40, font=\scriptsize},
    arrow/.style={->, thick, draw=black!40}
]
\node[faded] (1) {Patterning};
\node[highlight, below=0.5cm of 1] (2) {Etching};
\node[faded, below=0.5cm of 2] (3) {Post etching treatment};
\node[faded, below=of 3] (4) {Applying surfactant (DMOAP)};
\node[faded, below=of 4] (5) {Cell assembly};
\node[faded, below=of 5] (6) {Cell thickness measurement};
\node[faded, below=of 6] (7) {LC filling and cooling};

\draw[arrow] (1) -- (2);
\draw[arrow] (2) -- (3);
\draw[arrow] (3) -- (4);
\draw[arrow] (4) -- (5);
\draw[arrow] (5) -- (6);
\draw[arrow] (6) -- (7);
\end{tikzpicture}

\end{column}

\begin{column}{0.75\textwidth}
Here you explain what happens at stage C.
You can add equations, figures, or intuition here.
\end{column}
\end{columns}

\end{frame}

% Post etching--------------------------------------------------------------------------------------
\begin{frame}{\small{Preparing Homeotropic Cell}}

\begin{columns}

\begin{column}{0.25\textwidth}

\centering
\begin{tikzpicture}[
    node distance=0.25cm,
    every node/.style={draw, rectangle, rounded corners, minimum width=0.5cm, minimum height=0.5cm, text width=2.7cm, align=center},
    highlight/.style={draw=black, thick, fill=black!15},
    faded/.style={draw=black!40, text=black!40, font=\scriptsize},
    arrow/.style={->, thick, draw=black!40}
]
\node[faded] (1) {Patterning};
\node[faded, below=of 1] (2) {Etching};
\node[highlight, below=0.5cm of 2] (3) {Post etching treatment};
\node[faded, below=0.5cm of 3] (4) {Applying surfactant (DMOAP)};
\node[faded, below=of 4] (5) {Cell assembly};
\node[faded, below=of 5] (6) {Cell thickness measurement};
\node[faded, below=of 6] (7) {LC filling and cooling};

\draw[arrow] (1) -- (2);
\draw[arrow] (2) -- (3);
\draw[arrow] (3) -- (4);
\draw[arrow] (4) -- (5);
\draw[arrow] (5) -- (6);
\draw[arrow] (6) -- (7);
\end{tikzpicture}

\end{column}

\begin{column}{0.75\textwidth}
Here you explain what happens at stage C.
You can add equations, figures, or intuition here.
\end{column}
\end{columns}

\end{frame}

% DMOAP---------------------------------------------------------------------------------------------
\begin{frame}{\small{Preparing Homeotropic Cell}}

\begin{columns}

\begin{column}{0.25\textwidth}

\centering
\begin{tikzpicture}[
    node distance=0.25cm,
    every node/.style={draw, rectangle, rounded corners, minimum width=0.5cm, minimum height=0.5cm, text width=2.7cm, align=center},
    highlight/.style={draw=black, thick, fill=black!15},
    faded/.style={draw=black!40, text=black!40, font=\scriptsize},
    arrow/.style={->, thick, draw=black!40}
]
\node[faded] (1) {Patterning};
\node[faded, below=of 1] (2) {Etching};
\node[faded, below=of 2] (3) {Post etching treatment};
\node[highlight, below=0.5cm of 3] (4) {Applying surfactant (DMOAP)};
\node[faded, below=0.5cm of 4] (5) {Cell assembly};
\node[faded, below=of 5] (6) {Cell thickness measurement};
\node[faded, below=of 6] (7) {LC filling and cooling};

\draw[arrow] (1) -- (2);
\draw[arrow] (2) -- (3);
\draw[arrow] (3) -- (4);
\draw[arrow] (4) -- (5);
\draw[arrow] (5) -- (6);
\draw[arrow] (6) -- (7);
\end{tikzpicture}

\end{column}

\begin{column}{0.75\textwidth}
Here you explain what happens at stage C.
You can add equations, figures, or intuition here.
\end{column}
\end{columns}

\end{frame}

% Cell assembly-------------------------------------------------------------------------------------
\begin{frame}{\small{Preparing Homeotropic Cell}}

\begin{columns}

\begin{column}{0.25\textwidth}

\centering
\begin{tikzpicture}[
    node distance=0.25cm,
    every node/.style={draw, rectangle, rounded corners, minimum width=0.5cm, minimum height=0.5cm, text width=2.7cm, align=center},
    highlight/.style={draw=black, thick, fill=black!15},
    faded/.style={draw=black!40, text=black!40, font=\scriptsize},
    arrow/.style={->, thick, draw=black!40}
]
\node[faded] (1) {Patterning};
\node[faded, below=of 1] (2) {Etching};
\node[faded, below=of 2] (3) {Post etching treatment};
\node[faded, below=of 3] (4) {Applying surfactant (DMOAP)};
\node[highlight, below=0.5cm of 4] (5) {Cell assembly};
\node[faded, below=0.5cm of 5] (6) {Cell thickness measurement};
\node[faded, below=of 6] (7) {LC filling and cooling};

\draw[arrow] (1) -- (2);
\draw[arrow] (2) -- (3);
\draw[arrow] (3) -- (4);
\draw[arrow] (4) -- (5);
\draw[arrow] (5) -- (6);
\draw[arrow] (6) -- (7);
\end{tikzpicture}

\end{column}

\begin{column}{0.75\textwidth}
Here you explain what happens at stage C.
You can add equations, figures, or intuition here.
\end{column}
\end{columns}

\end{frame}

% Cell thickness------------------------------------------------------------------------------------
\begin{frame}{\small{Preparing Homeotropic Cell}}

\begin{columns}

\begin{column}{0.25\textwidth}

\centering
\begin{tikzpicture}[
    node distance=0.25cm,
    every node/.style={draw, rectangle, rounded corners, minimum width=0.5cm, minimum height=0.5cm, text width=2.7cm, align=center},
    highlight/.style={draw=black, thick, fill=black!15},
    faded/.style={draw=black!40, text=black!40, font=\scriptsize},
    arrow/.style={->, thick, draw=black!40}
]
\node[faded] (1) {Patterning};
\node[faded, below=of 1] (2) {Etching};
\node[faded, below=of 2] (3) {Post etching treatment};
\node[faded, below=of 3] (4) {Applying surfactant (DMOAP)};
\node[faded, below=of 4] (5) {Cell assembly};
\node[highlight, below=0.5cm of 5] (6) {Cell thickness measurement};
\node[faded, below=0.5cm of 6] (7) {LC filling and cooling};

\draw[arrow] (1) -- (2);
\draw[arrow] (2) -- (3);
\draw[arrow] (3) -- (4);
\draw[arrow] (4) -- (5);
\draw[arrow] (5) -- (6);
\draw[arrow] (6) -- (7);
\end{tikzpicture}

\end{column}

\begin{column}{0.75\textwidth}
Here you explain what happens at stage C.
You can add equations, figures, or intuition here.
\end{column}
\end{columns}

\end{frame}

% Cell thickness------------------------------------------------------------------------------------
\begin{frame}{\small{Preparing Homeotropic Cell}}

\begin{columns}

\begin{column}{0.25\textwidth}

\centering
\begin{tikzpicture}[
    node distance=0.25cm,
    every node/.style={draw, rectangle, rounded corners, minimum width=0.5cm, minimum height=0.5cm, text width=2.7cm, align=center},
    highlight/.style={draw=black, thick, fill=black!15},
    faded/.style={draw=black!40, text=black!40, font=\scriptsize},
    arrow/.style={->, thick, draw=black!40}
]
\node[faded] (1) {Patterning};
\node[faded, below=of 1] (2) {Etching};
\node[faded, below=of 2] (3) {Post etching treatment};
\node[faded, below=of 3] (4) {Applying surfactant (DMOAP)};
\node[faded, below=of 4] (5) {Cell assembly};
\node[faded, below=of 5] (6) {Cell thickness measurement};
\node[highlight, below=0.5cm of 6] (7) {LC filling and cooling};

\draw[arrow] (1) -- (2);
\draw[arrow] (2) -- (3);
\draw[arrow] (3) -- (4);
\draw[arrow] (4) -- (5);
\draw[arrow] (5) -- (6);
\draw[arrow] (6) -- (7);
\end{tikzpicture}

\end{column}

\begin{column}{0.75\textwidth}
Here you explain what happens at stage C.
You can add equations, figures, or intuition here.
\end{column}
\end{columns}

\end{frame}

\begin{frame}{\small{Optical microscopy system}}
    
\end{frame}

\section{Computational Framework}

\begin{frame}{Computational Framework}
\begin{columns}

\begin{column}{0.25\textwidth}

\centering
\begin{tikzpicture}[
    node distance=0.25cm,
    every node/.style={draw, rectangle, rounded corners, minimum width=0.5cm, minimum height=0.5cm, text width=2.7cm, align=center},
    highlight/.style={draw=black, thick, fill=black!15, font=\small},
    faded/.style={draw=black!40, text=black!40, font=\scriptsize},
    arrow/.style={->, thick, draw=black!40}
]
\node[highlight] (1) {Pre-processing and noise reduction};
\node[highlight, below=of 1] (2) {Ridge detection};
\node[highlight, below=of 2] (3) {Hysteresis thresholding};
\node[highlight, below=of 3] (4) {Binary closing and cleaning};
\node[highlight, below=of 4] (5) {Skeletonizing};
\node[highlight, below=of 5] (6) {Estimating defect density};

\draw[arrow] (1) -- (2);
\draw[arrow] (2) -- (3);
\draw[arrow] (3) -- (4);
\draw[arrow] (4) -- (5);
\draw[arrow] (5) -- (6);

\end{tikzpicture}

\end{column}

\begin{column}{0.75\textwidth}
Here you explain what happens at stage C.
You can add equations, figures, or intuition here.\\
\vspace{\fill}
\begin{figure}[b]
    \centering
    \includegraphics[width=1\linewidth]{figs/test-framework-1_50s_28_60V_5-S012.png}
    \caption{Caption}
    \label{fig:placeholder}
\end{figure}
\end{column}
\end{columns}

\end{frame}

% Noise reduction-----------------------------------------------------------------------------------
\begin{frame}{Computational Framework}
\begin{columns}

\begin{column}{0.25\textwidth}

\centering
\begin{tikzpicture}[
    node distance=0.25cm,
    every node/.style={draw, rectangle, rounded corners, minimum width=0.5cm, minimum height=0.5cm, text width=2.7cm, align=center},
    highlight/.style={draw=black, thick, fill=black!15},
    faded/.style={draw=black!40, text=black!40, font=\scriptsize},
    arrow/.style={->, thick, draw=black!40}
]
\node[highlight] (1) {Pre-processing and noise reduction};
\node[faded, below=0.5cm of 1] (2) {Ridge detection};
\node[faded, below=of 2] (3) {Hysteresis thresholding};
\node[faded, below=of 3] (4) {Binary closing and cleaning};
\node[faded, below=of 4] (5) {Skeletonizing};
\node[faded, below=of 5] (6) {Estimating defect density};

\draw[arrow] (1) -- (2);
\draw[arrow] (2) -- (3);
\draw[arrow] (3) -- (4);
\draw[arrow] (4) -- (5);
\draw[arrow] (5) -- (6);

\end{tikzpicture}

\end{column}

\begin{column}{0.75\textwidth}
Here you explain what happens at stage C.
You can add equations, figures, or intuition here.\\
\vspace{\fill}
\begin{figure}[b]
    \centering
    \includegraphics[width=1\linewidth]{figs/test-framework-1_50s_28_60V_5-S012.png}
    \caption{Caption}
    \label{fig:placeholder}
\end{figure}
\end{column}
\end{columns}

\end{frame}

% Ridge detection-----------------------------------------------------------------------------------
\begin{frame}{Computational Framework}
\begin{columns}

\begin{column}{0.25\textwidth}

\centering
\begin{tikzpicture}[
    node distance=0.25cm,
    every node/.style={draw, rectangle, rounded corners, minimum width=0.5cm, minimum height=0.5cm, text width=2.7cm, align=center},
    highlight/.style={draw=black, thick, fill=black!15},
    faded/.style={draw=black!40, text=black!40, font=\scriptsize},
    arrow/.style={->, thick, draw=black!40}
]
\node[faded] (1) {Pre-processing and noise reduction};
\node[highlight, below=0.5cm of 1] (2) {Ridge detection};
\node[faded, below=0.5cm of 2] (3) {Hysteresis thresholding};
\node[faded, below=of 3] (4) {Binary closing and cleaning};
\node[faded, below=of 4] (5) {Skeletonizing};
\node[faded, below=of 5] (6) {Estimating defect density};

\draw[arrow] (1) -- (2);
\draw[arrow] (2) -- (3);
\draw[arrow] (3) -- (4);
\draw[arrow] (4) -- (5);
\draw[arrow] (5) -- (6);

\end{tikzpicture}

\end{column}

\begin{column}{0.75\textwidth}
Here you explain what happens at stage C.
You can add equations, figures, or intuition here.\\
\vspace{\fill}
\begin{figure}[b]
    \centering
    \includegraphics[width=1\linewidth]{figs/test-framework-1_50s_28_60V_5-S012.png}
    \caption{Caption}
    \label{fig:placeholder}
\end{figure}
\end{column}
\end{columns}

\end{frame}

% Hysteresis thresholding---------------------------------------------------------------------------
\begin{frame}{Computational Framework}
\begin{columns}

\begin{column}{0.25\textwidth}

\centering
\begin{tikzpicture}[
    node distance=0.25cm,
    every node/.style={draw, rectangle, rounded corners, minimum width=0.5cm, minimum height=0.5cm, text width=2.7cm, align=center},
    highlight/.style={draw=black, thick, fill=black!15},
    faded/.style={draw=black!40, text=black!40, font=\scriptsize},
    arrow/.style={->, thick, draw=black!40}
]
\node[faded] (1) {Pre-processing and noise reduction};
\node[faded, below=of 1] (2) {Ridge detection};
\node[highlight, below=0.5cm of 2] (3) {Hysteresis thresholding};
\node[faded, below=0.5cm of 3] (4) {Binary closing and cleaning};
\node[faded, below=of 4] (5) {Skeletonizing};
\node[faded, below=of 5] (6) {Estimating defect density};

\draw[arrow] (1) -- (2);
\draw[arrow] (2) -- (3);
\draw[arrow] (3) -- (4);
\draw[arrow] (4) -- (5);
\draw[arrow] (5) -- (6);

\end{tikzpicture}

\end{column}

\begin{column}{0.75\textwidth}
Here you explain what happens at stage C.
You can add equations, figures, or intuition here.\\
\vspace{\fill}
\begin{figure}[b]
    \centering
    \includegraphics[width=1\linewidth]{figs/test-framework-1_50s_28_60V_5-S345.png}
    \caption{Caption}
    \label{fig:placeholder}
\end{figure}
\end{column}
\end{columns}

\end{frame}

% Binary closing and cleaning-----------------------------------------------------------------------
\begin{frame}{Computational Framework}
\begin{columns}

\begin{column}{0.25\textwidth}

\centering
\begin{tikzpicture}[
    node distance=0.25cm,
    every node/.style={draw, rectangle, rounded corners, minimum width=0.5cm, minimum height=0.5cm, text width=2.7cm, align=center},
    highlight/.style={draw=black, thick, fill=black!15},
    faded/.style={draw=black!40, text=black!40, font=\scriptsize},
    arrow/.style={->, thick, draw=black!40}
]
\node[faded] (1) {Pre-processing and noise reduction};
\node[faded, below=of 1] (2) {Ridge detection};
\node[faded, below=of 2] (3) {Hysteresis thresholding};
\node[highlight, below=0.5cm of 3] (4) {Binary closing and cleaning};
\node[faded, below=0.5cm of 4] (5) {Skeletonizing};
\node[faded, below=of 5] (6) {Estimating defect density};

\draw[arrow] (1) -- (2);
\draw[arrow] (2) -- (3);
\draw[arrow] (3) -- (4);
\draw[arrow] (4) -- (5);
\draw[arrow] (5) -- (6);

\end{tikzpicture}

\end{column}

\begin{column}{0.75\textwidth}
Here you explain what happens at stage C.
You can add equations, figures, or intuition here.\\
\vspace{\fill}
\begin{figure}[b]
    \centering
    \includegraphics[width=1\linewidth]{figs/test-framework-1_50s_28_60V_5-S345.png}
    \caption{Caption}
    \label{fig:placeholder}
\end{figure}
\end{column}
\end{columns}

\end{frame}

% Skletonizing--------------------------------------------------------------------------------------
\begin{frame}{Computational Framework}
\begin{columns}

\begin{column}{0.25\textwidth}

\centering
\begin{tikzpicture}[
    node distance=0.25cm,
    every node/.style={draw, rectangle, rounded corners, minimum width=0.5cm, minimum height=0.5cm, text width=2.7cm, align=center},
    highlight/.style={draw=black, thick, fill=black!15},
    faded/.style={draw=black!40, text=black!40, font=\scriptsize},
    arrow/.style={->, thick, draw=black!40}
]
\node[faded] (1) {Pre-processing and noise reduction};
\node[faded, below=of 1] (2) {Ridge detection};
\node[faded, below=of 2] (3) {Hysteresis thresholding};
\node[faded, below=of 3] (4) {Binary closing and cleaning};
\node[highlight, below=0.5cm of 4] (5) {Skeletonizing};
\node[faded, below=0.5cm of 5] (6) {Estimating defect density};

\draw[arrow] (1) -- (2);
\draw[arrow] (2) -- (3);
\draw[arrow] (3) -- (4);
\draw[arrow] (4) -- (5);
\draw[arrow] (5) -- (6);

\end{tikzpicture}

\end{column}

\begin{column}{0.75\textwidth}
Here you explain what happens at stage C.
You can add equations, figures, or intuition here.\\
\vspace{\fill}
\begin{figure}[b]
    \centering
    \includegraphics[width=1\linewidth]{figs/test-framework-1_50s_28_60V_5-S345.png}
    \caption{Caption}
    \label{fig:placeholder}
\end{figure}
\end{column}
\end{columns}

\end{frame}

% Defect density------------------------------------------------------------------------------------
\begin{frame}{Computational Framework}
\begin{columns}

\begin{column}{0.25\textwidth}

\centering
\begin{tikzpicture}[
    node distance=0.25cm,
    every node/.style={draw, rectangle, rounded corners, minimum width=0.5cm, minimum height=0.5cm, text width=2.7cm, align=center},
    highlight/.style={draw=black, thick, fill=black!15},
    faded/.style={draw=black!40, text=black!40, font=\scriptsize},
    arrow/.style={->, thick, draw=black!40}
]
\node[faded] (1) {Pre-processing and noise reduction};
\node[faded, below=of 1] (2) {Ridge detection};
\node[faded, below=of 2] (3) {Hysteresis thresholding};
\node[faded, below=of 3] (4) {Binary closing and cleaning};
\node[faded, below=of 4] (5) {Skeletonizing};
\node[highlight, below=0.5cm of 5] (6) {Estimating defect density};

\draw[arrow] (1) -- (2);
\draw[arrow] (2) -- (3);
\draw[arrow] (3) -- (4);
\draw[arrow] (4) -- (5);
\draw[arrow] (5) -- (6);

\end{tikzpicture}

\end{column}

\begin{column}{0.75\textwidth}
Here you explain what happens at stage C.
You can add equations, figures, or intuition here.\\


\end{column}
\end{columns}

\end{frame}
\begin{frame}{\small{Error Analysis}}

    
\end{frame}

\section{Defect Dynamics}
\begin{frame}{Late time dynamics}
    
\end{frame}

\begin{frame}{Formation dynamics}
    
\end{frame}

\section{Conclusion}
\begin{frame}[fragile]{Conclusion}
hk

\end{frame}

\begin{frame}{Acknowledgement}

\end{frame}

\end{document}